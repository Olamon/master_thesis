\subsection{Motivation} Motivation for studying this model can be found in 
IP-based computer networks problems. A router has to store an enormous number of
forwarding rules, and this number is still growing. The fast router memory is
expensive and requires a lot of energy, which is a big problem for Internet
Service Providers. As a solution to this problem, there comes Software-Defined
Networking (SDN) technology. It consists of two types of memory: a fast memory,
which is kept in the router and a slow one, which is placed in a so called controller.
The latter keeps exact information about all of the forwarding rules (the
forwarding rules are called Forwarding Information Base and the abbreviation FIB
is commonly used).  On the other hand, the router keeps only the part of the FIB
table. Whenever a packet comes, it is firstly processed by the router. If its
forwarding rule is found in the router's cache, then it is immediately forwarded
to its final destination. In the case where the appropriate rule is not found for
the packet at the router, we have to ask the controller for it. We therefore look
up the rule in the slow memory and pay $1$ for each such request. The router's
cache can change its state, but this operation is usually expensive. Precisely,
if we decide to erase a single rule form the router or fetch it from the
controller to the router, we pay $\alpha$ for each such opertion. Moreover,
occasionaly an update might take place at some rule in the controller's FIB table.
If the corresponting rule is stored in the router, we are obligated to make the
update to it immediately, which costs us $\alpha$, as well. Figure
\ref{fig:motivation} shows the crucial concepts that were just briefly
described. A more detailed and technical description can found in \cite{sdn}.
\begin{figure} \begin{center}
\includegraphics[width=0.8\textwidth]{motivation.png} \end{center} \caption{}
\label{fig:motivation} \end{figure}

Since the rules are taken from the forwarding table, what we look for is the
rule with the longest matching prefix (LMP). Whenever a packet arrives, the
router will look for the rule which matches the longest prefix of the
destination IP address of the packet. Consequently, the search can be viewed as
going down a tree whose nodes hold IP address prefixes and whose
root keeps the least specific (empty) rule. Additionally, the descendants of a node
representing IP prefix $p$ are the nodes with rules $p \circ x$, where the operation
$\circ$ is the concatenation of binary representations of $p$ and $x$. The deeper in
the tree we can go, the more specific of a rule we will find. We add a special rule,
which is kept in the root of router's cache tree, that matches every prefix and
points to the controller. Hence, we ask the
controller if actual matching rule can't be found in the router's cache. Note that besides
the tree-like structure that the router's cache
consists of, there is natural requirement of the router's tree to be
bottom-contiguous. Otherwise, in some cases, we would not find the most specific
rule for the packet and therefore we would forward the packet to the wrong addess. 

Summing up, we can define a model for SDN architecture, which from now on will
be called the SDN model. In that model, as in the tree caching model, we have a tree
$T$ whose nodes are requested, fast memory (cache) that can store $\kind{ONL}$
elements (which are the forwarding rules), which form a bottom-contiguous forest,
and the slow memory keeping the whole tree structure. The input consists of
requests for the elements of the tree. We can serve cached items for free by finding
the appropriate rule in the cache, whereas the ones that match only special rule
(empty) require asking the controller, and thus we pay $1$. Any change in the
cache costs us $\alpha$, the same as in the tree caching model. We have an
additional event though, because the FIB rules may change. Whenever a controller
rule is updated, we update the corresponding rule in the router's cache paying
$\alpha$ if it is present in the router's cache, and otherwise we do not perform
an update and do not have to pay anything. We call such event \emph{rule update}.

Now we will use an algorithm $TC$ to solve the tree caching problem as a black
box and adopt it to solve the SDN setting (by defining algorithm $ALG$ based on
$TC$). First, the tree from the SDN model is mapped to the tree caching model's
tree using the identity function. Let $\sigma$ be the input for the SDN model. We
transform it to $\sigma'$, which will be the input for $TC$. We do so by adding
$\alpha$ negative requests to a rule each time it was updated in the
controller (recall that in the SDN model in such a situation we are obtaining
cost $\alpha$ for the update if the rule was in cache). The rest of the input
stays the same with only one change to make all of these requests positive. Now
$ALG$ is defined in such a way that given input sequence $\sigma$, it copies the
moves (fetches and evictions) of $TC$ (operating on the  modified input $\sigma'$). The
following theorem states that using the described application of $TC$, we can
get an algorithm with the same competitive ratio for the SDN model as for the
tree caching model up to the constant factor. Therefore, solving the tree
caching problem gives us a competitive algorithm for the SDN model, and thus 
our model has a very useful practical application.  \begin{theorem} Let
$TC$ be any algorithm solving the tree caching problem that is $r$-competitive.
Let $ALG$, $\sigma$ and $\sigma'$ be defined as above with respect to $TC$.
Then: $$cost_{\mathrm{ALG}}(\sigma) \leq c \cdot cost_{\mathrm{TC}}(\sigma')
\leq c \cdot r \cdot cost_{\mathrm{OPT'}}(\sigma') \leq c \cdot r \cdot
cost_{\mathrm{OPT}}(\sigma),$$ where $c$ is a constant, $OPT'$ is the
optimal offline solution for the tree caching model, and $OPT$ is the
optimal offline solution for the SDN model. In other words, $ALG$ has the same
competitive ratio as $TC$ up to the constant factor.  \end{theorem}
\begin{proof} We get the second inequality directly from the property of $TC$
being $r$-competitive. To prove the first inequality, we have to compare the
costs of $ALG$ on $\sigma$ with the cost of $TC$ on $\sigma'$. Let $f$ be the
mapping from requests in the sequence $\sigma$ to the positive requests of
$\sigma'$ that preserves the order of events. To any update to the 
rule $u$ in the controller, we assign $m(u)$, which is a set of $\alpha$ negative
requests that were added to the sequence $\sigma'$ because of that update. There
are three cases when $ALG$ has to pay. \begin{enumerate} \item $ALG$ processes
the request $\sigma_t$ to an element that is not present in the $ALG$'s cache.
Then it pays $1$ for serving that element. $TC$ has the same cache state when it
processes the corresponding positive request $f(\sigma_t)$, thus paying $1$ as well.
The costs of both algorithms are equal in this case. \item $ALG$ makes a
reorganization of the cache. Recall that it makes the same reorganizations as
$TC$, thus the overall cost related to changing the cache is the same for both
algorithms. \item Update $u$ takes place for some rule in the controller, so
$ALG$ pays $\alpha$. At the same time we have a sequence of $\alpha$ negative
requests $m(u)$. $TC$ pays then at least $\alpha$ on that sequence of negative requests - either for
all the requests or the cache update. 
\end{enumerate}
$ALG$ and $TC$ both do not pay for any
other events besides the ones that were listed above. The cost of each event for $TC$ was used
at most twice (look at the last point above, we compare $ALG$ with the cache
reorganization cost of $TC$ which was already used in the second point), and
therefore $cost_{\mathrm{ALG}}(\sigma)
\leq 2 \cdot cost_{\mathrm{TC}}(\sigma')$, which ends the proof of the first inequality.

Now we proceed to the proof of the last inequality. Consider an
offline algorithm $OFF$ for the tree caching problem that makes the same moves as
the $OPT$ on $\sigma$. It does not pay on the rule updates, therefore the cost
incurred on the negative requests can be compensated by the updates, so
$cost_{\mathrm{OFF}}(\sigma') \leq cost_{\mathrm{OPT}}(\sigma)$. Obviously,
$cost_{\mathrm{OPT'}}(\sigma')
\leq cost_{\mathrm{OFF}}(\sigma')$, which ends the proof. \end{proof}
