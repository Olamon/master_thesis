\begin{figure}
\begin{postscript}
% \usepackage[usenames,dvipsnames]{pstricks}
% \usepackage{epsfig}
% \usepackage{pst-grad} % For gradients
% \usepackage{pst-plot} % For axes
\psscalebox{0.75 0.75} % Change this value to rescale the drawing.
{
\begin{pspicture}(0,-2.0524511)(16.918556,2.0524511)
\definecolor{colour0}{rgb}{0.78431374,0.78431374,0.78431374}
\psframe[linecolor=black, linewidth=0.04, 
dimen=outer](15.62,1.2324512)(0.02,0.03245117)
\psframe[linecolor=black, linewidth=0.04, 
dimen=outer](1.22,1.2324512)(0.02,0.03245117)
\psframe[linecolor=black, linewidth=0.04, 
dimen=outer](2.42,1.2324512)(1.22,0.03245117)
\psframe[linecolor=black, linewidth=0.04, 
dimen=outer](3.62,1.2324512)(2.42,0.03245117)
\psframe[linecolor=black, linewidth=0.04, 
dimen=outer](4.82,1.2324512)(3.62,0.03245117)
\psframe[linecolor=black, linewidth=0.04, 
dimen=outer](6.02,1.2324512)(4.82,0.03245117)
\psframe[linecolor=black, linewidth=0.04, fillstyle=solid,fillcolor=colour0, 
dimen=outer](7.22,1.2324512)(6.02,0.03245117)
\psframe[linecolor=black, linewidth=0.04, fillstyle=hlines*,fillcolor=colour0, 
hatchwidth=0.028222222, hatchangle=0.0, hatchsep=0.2412, 
dimen=outer](8.42,1.2324512)(7.22,0.03245117)
\psframe[linecolor=black, linewidth=0.04, fillstyle=solid,fillcolor=colour0, 
dimen=outer](9.62,1.2324512)(8.42,0.03245117)
\psframe[linecolor=black, linewidth=0.04, fillstyle=hlines*,fillcolor=colour0,
hatchwidth=0.028222222, hatchangle=0.0, hatchsep=0.2412, 
dimen=outer](10.82,1.2324512)(9.62,0.03245117)
\psframe[linecolor=black, linewidth=0.04, 
dimen=outer](12.02,1.2324512)(10.82,0.03245117)
\psframe[linecolor=black, linewidth=0.04, fillstyle=hlines, 
hatchwidth=0.028222222, hatchangle=0.0, hatchsep=0.2412, 
dimen=outer](13.22,1.2324512)(12.02,0.03245117)
\psframe[linecolor=black, linewidth=0.04, 
dimen=outer](14.42,1.2324512)(13.22,0.03245117)
\psframe[linecolor=black, linewidth=0.04, fillstyle=solid,fillcolor=colour0, 
dimen=outer](15.62,1.2324512)(14.42,0.03245117)
\psdots[linecolor=black, dotsize=0.2](16.02,0.03245117)
\psdots[linecolor=black, dotsize=0.2](16.42,0.03245117)
\psdots[linecolor=black, dotsize=0.2](16.82,0.03245117)
\psline[linecolor=black, 
linewidth=0.04](0.02,-0.36754882)(0.02,-0.7675488)(6.02,-0.7675488)(6.02,
-0.36754882)(6.02,-0.36754882)
\psline[linecolor=black, 
linewidth=0.04](6.02,1.6324512)(6.02,2.0324512)(10.82,2.0324512)(10.82,
1.6324512)(10.82,1.6324512)
\psline[linecolor=black, 
linewidth=0.04](10.82,-0.36754882)(10.82,-0.7675488)(14.42,-0.7675488)(14.42,
-0.36754882)(14.42,-0.36754882)
\psline[linecolor=black, 
linewidth=0.04](14.42,1.6324512)(14.42,2.0324512)(16.02,2.0324512)(16.02,
2.0324512)
\rput(2.82,-0.36754882){PHASE 1}
\rput(8.42,1.6324512){PHASE 2}
\rput(12.82,-0.36754882){PHASE 3}
\rput[bl](0.42,0.43245116){1}
\rput[bl](1.62,0.43245116){2}
\rput[bl](2.82,0.43245116){1}
\rput[bl](4.02,0.43245116){3}
\rput[bl](5.22,0.43245116){2}
\rput[bl](6.42,0.43245116){4}
\rput[bl](7.62,0.43245116){2}
\rput[bl](8.82,0.43245116){5}
\rput[bl](10.02,0.43245116){2}
\rput[bl](11.22,0.43245116){3}
\rput[bl](12.42,0.43245116){4}
\rput[bl](13.62,0.43245116){1}
\rput[bl](14.82,0.43245116){5}
\rput[bl](6.42,-1.5675489){element k+1}
\psline[linecolor=black, linewidth=0.04, arrowsize=0.05291666666666668cm 
2.0,arrowlength=1.4,arrowinset=0.0]{<-}(6.42,0.43245116)(7.22,-1.1675488)(7.22,
-1.1675488)
\psline[linecolor=black, linewidth=0.04, arrowsize=0.05291666666666668cm 
2.0,arrowlength=1.4,arrowinset=0.0]{<-}(8.02,0.43245116)(10.42,-1.5675489)(10.42
,-1.5675489)
\psline[linecolor=black, linewidth=0.04, arrowsize=0.05291666666666668cm 
2.0,arrowlength=1.4,arrowinset=0.0]{<-}(10.42,0.43245116)(10.42,
-1.5675489)(10.42,-1.5675489)
\psline[linecolor=black, linewidth=0.04, arrowsize=0.05291666666666668cm 
2.0,arrowlength=1.4,arrowinset=0.0]{<-}(12.42,0.43245116)(10.42,
-1.5675489)(10.42,-1.5675489)
\rput[bl](9.62,-1.9675488){old elements}
\end{pspicture}
}
\end{postscript}
\caption{Example of sequence devided to phases for cache size $k=3$. Old pages 
are hatched. The number of new pages in fist phase $m_1 = 3$ and for second and 
third phase $m_2 = m_3 = 2$.}
\label{fig:PhasesRandomized}
\end{figure}