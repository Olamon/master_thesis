\section{Tree caching with dependencies}
The problem we consider (called \textit{tree caching problem})is an extension 
of classic caching problem with bypassing that was recalled in section 2. The 
main difference is that requested items have inter-dependencies. Specifically, 
universe forms rooted tree $T$ and the items requested are placed in its nodes 
and whenever $v \in T$ is fetched all of its descendants $T(v)$ that are not in 
cache have to be fetched at the same time. We say that the cache with property 
described above is \textit{bottom-contiguous}.

Moreover, we consider two types of requests: \textit{negative} and 
\textit{positive}. If incoming request is positive we pay 1 if it is 
not present in the cache (so the same as in most considered settings). On the 
other hand we pay for negative requests if the item is cached (so in opposite 
situation). After serving of any request we can reorganize cache paying 
$\alpha \geq 1$ for every element evicted or fetched. 

The goal is to minimize the cost obtained by online algorithm solving this 
problem. We present a deterministic algorithm \textbf{TRC} which we prove is 
$O(h(T) \times R)$-competitive where $h(T)$ is height of input tree and $R = 
k_{ONL}/(k_{ONL} - k_{OPT} +1$ with $k_{ONL}$ and $k_{OPT}$ being size of 
online and optimal offline algorithm cache.

\subsection{Model}
\begin{figure}
\begin{center}
\begin{postscript}

\psscalebox{0.6 0.6} % Change this value to rescale the drawing.
{
\begin{pspicture}(0,-4.0099998)(12.562361,4.0099998)
\definecolor{colour0}{rgb}{0.78431374,0.78431374,0.78431374}
\psbezier[linecolor=black, linewidth=0.04, 
fillstyle=solid,fillcolor=colour0](4.8323607,3.9899998)(3.8323605,
3.9899998)(0.046042196,0.2078214)(0.8323605,-0.41000015)(1.6186789,
-1.0278217)(9.759687,-1.1499403)(10.432361,-0.41000015)(11.105033,
0.32993993)(5.8323607,3.9899998)(4.8323607,3.9899998)
\pscircle[linecolor=black, linewidth=0.04, fillstyle=solid,fillcolor=black, 
dimen=outer](4.8323607,2.79){0.4}
\pscircle[linecolor=black, linewidth=0.04, fillstyle=solid,fillcolor=black, 
dimen=outer](3.2323606,1.9899999){0.4}
\pscircle[linecolor=black, linewidth=0.04, fillstyle=solid,fillcolor=black, 
dimen=outer](4.8323607,1.1899998){0.4}
\pscircle[linecolor=black, linewidth=0.04, fillstyle=solid,fillcolor=black, 
dimen=outer](6.4323606,1.9899999){0.4}
\pscircle[linecolor=black, linewidth=0.04, fillstyle=solid,fillcolor=black, 
dimen=outer](8.03236,1.1899998){0.4}
\pscircle[linecolor=black, linewidth=0.04, fillstyle=solid,fillcolor=black, 
dimen=outer](7.6323605,-0.41000015){0.4}
\pscircle[linecolor=black, linewidth=0.04, fillstyle=solid,fillcolor=black, 
dimen=outer](9.232361,-0.0100001525){0.4}
\psline[linecolor=black, linewidth=0.04](4.8323607,2.79)(3.2323606,1.9899999)
\psline[linecolor=black, linewidth=0.04](4.8323607,2.79)(4.8323607,1.1899998)
\psline[linecolor=black, 
linewidth=0.04](4.8323607,2.79)(6.4323606,1.9899999)(6.4323606,1.9899999)
\psline[linecolor=black, 
linewidth=0.04](6.4323606,1.9899999)(8.03236,1.1899998)(7.6323605,
-0.41000015)(8.03236,1.1899998)(9.232361,-0.0100001525)(9.232361,-0.41000015)
\psline[linecolor=black, 
linewidth=0.04](3.2323606,1.9899999)(3.2323606,1.9899999)
\psline[linecolor=black, 
linewidth=0.04](4.8323607,2.79)(4.8323607,2.79)(4.8323607,2.79)
\pstriangle[linecolor=black, linewidth=0.06, 
dimen=outer](1.2323605,-3.6100001)(2.4,2.8)
\pstriangle[linecolor=black, linewidth=0.06, 
dimen=outer](4.8323607,-3.6100001)(2.4,2.8)
\psline[linecolor=black, 
linewidth=0.04](4.8323607,0.78999984)(4.8323607,-0.8100002)(4.8323607,
-0.8100002)
\pstriangle[linecolor=black, linewidth=0.06, 
dimen=outer](9.232361,-3.6100001)(2.4,2.8)
\psline[linecolor=black, 
linewidth=0.04](2.8323605,-0.0100001525)(2.8323605,-0.0100001525)(2.8323605,
-0.0100001525)(2.8323605,-0.0100001525)
\rput(11.63236,1.9899999){Tree cap}
\rput(9.232361,-0.0100001525){\textcolor{white}{\textbf{r}}}
\pspolygon[linecolor=black, linewidth=0.04, fillstyle=vlines, 
hatchwidth=0.028222222, hatchangle=0.0, 
hatchsep=0.1411111](2.4323606,-3.6100001)(1.6323606,-2.41)(1.2323605,
-2.8100002)(0.8323605,-2.0100002)(0.032360535,-3.6100001)(0.032360535,
-3.6100001)
\pspolygon[linecolor=black, linewidth=0.04, fillstyle=vlines, 
hatchwidth=0.028222222, hatchangle=0.0, 
hatchsep=0.1411111](4.4323606,-3.6100001)(4.8323607,-2.41)(6.0323606,
-3.6100001)(6.0323606,-3.6100001)
\psframe[linecolor=black, linewidth=0.04, fillstyle=vlines, 
hatchwidth=0.028222222, hatchangle=0.0, hatchsep=0.1411111, 
dimen=outer](10.432361,3.1899998)(9.63236,2.3899999)
\rput(11.63236,2.79){Valid cache }
\psline[linecolor=black, 
linewidth=0.04](9.232361,-1.6100001)(9.232361,-1.6100001)(9.232361,
-1.6100001)(9.232361,-1.6100001)
\psline[linecolor=black, 
linewidth=0.04](9.232361,-0.0100001525)(9.232361,-0.8100002)(9.232361,
-0.8100002)
\pstriangle[linecolor=black, linewidth=0.06, linestyle=dotted, 
dotsep=0.10583334cm, fillstyle=vlines, hatchwidth=0.02, hatchangle=0.0, 
hatchsep=0.2212, dimen=outer](9.232361,-4.01)(4.0,5.2)
\psline[linecolor=black, 
linewidth=0.04](3.2323606,1.9899999)(1.2323605,-0.8100002)(1.2323605,-0.8100002)
\psline[linecolor=black, 
linewidth=0.04](3.2323606,1.1899998)(3.2323606,1.1899998)
\psline[linecolor=black, 
linewidth=0.04](4.8323607,3.59)(4.8323607,3.59)(4.8323607,3.59)(4.8323607,3.59)
\psframe[linecolor=black, linewidth=0.04, fillstyle=solid,fillcolor=colour0, 
dimen=outer](10.432361,2.3899999)(9.63236,1.5899998)
\psframe[linecolor=black, linewidth=0.064, linestyle=dotted, 
dotsep=0.10583334cm, fillstyle=solid, 
dimen=outer](10.432361,3.9899998)(9.63236,3.1899998)
\rput(11.63236,3.59){T(r)}
\end{pspicture}
}

\end{postscript}
\end{center}
\caption{Visualization of definitions for the tree caching problem.}
\label{fig:TreeCacheDefinitions}
\end{figure}

For ease of future algorithm description and readability of proofs we introduce 
some useful definitions. First of all by $T(r)$ we mean subtree of $T$ rooted 
in $r$ and by $h(T)$ height of tree. \textit{Tree cap} is a tree $T_c \subseteq 
T$ such that root $r$ of $T$ belongs to it and $v \in T_c$ implies that all the 
nodes on the path from $v$ to $r$ belong to $T_c$. We say that set $C \subseteq 
T$ is \textit{bottom contiguous} if $v \in C$ implies $T(v) \subseteq C$ and we 
say that set $C$ \textit{valid cache state}. Notice that it is not dependent on 
cache size so $C$ ca be valid cache state and not fit into cache size the same 
time.

The \textbf{TRC} algorithm, after each request, can change the state of cache, 
but the only possible changes have to leave the cache in valid state. 
Depending on the character of change (eviction vs. fetch) we can apply for cache 
$C$ either \textit{valid negative changeset} or \textit{valid positive 
changeset}. The valid positive changeset $X$ (think of it as a set that can be 
fetched) is a set for which $X \cap C = \varnothing$ and $C \cup X$ 
is a valid cache state, the negative one (related to eviction) is a set $X 
\subseteq C$ such that $C \ X$ is valid cache state.

We process the input one by one and assume each request $\sigma(t)$ comes in 
time interval $(t-1, t)$, so the next request comes in time interval $(t, t + 
1)$ and so on. At time $t$ algorithm changes cache state from $C_{t}$ to 
$C_{t+1}$. Cache state is well defined for round $t$ (which contains time 
interval $(t-1, t)$), because as we see the cache may change only between 
rounds.

Some of the definition which appeared in this section are presented on figure 
~\ref{fig:TreeCacheDefinitions}.

\subsection{Algorithm TRC}

Before finally defining the \textbf{TRC} algorithm we bind to each node $v \in 
T$ a $bank_{t}(v)$ value. At the beginning of algorithm for each $v \in T$ we 
have $bank_{0}(v) = 0$. If node $v$ gets negative or positive request at 
time $t$ for which we pay 1 we increase $bank_{t}(v) = bank_{t-1}(v) + 1$, 
leaving the value for any other node unchanged (so for $v' \neq v 
bank_{t-1}(v') = bank_{t}(v')$). When we change the state of node $v$ (either 
evict or fetch it into cache) we clear its bank value. For set of nodes $V 
\subseteq T$, we denote sum $\sum_{v \in V} bank_{t}(v)$ as $bank_{t}(V)$.

The \textbf{TRC} proceeds in phases. First phase starts at time 0 and at the 
start of each phase the \textbf{TRC} cache is empty. 
\begin{algorithm}
\caption{\textbf{TRC}}
\label{alg:TRC}
\begin{algorithmic}[1]
\ForAll{$t \in \{1 \ldots |\sigma|\}$}
  \State Serve request $\sigma(t)$.
  \State Compute $bank_t(v)$ values for $v \in T$.
  \If{exists valid changeset $X$ such that: \begin{itemize}
    \item $bank_t(X) \geq |X| \times \alpha$ (saturation),
    \item $bank_t(Y) < |Y| \times \alpha$ for any valid changeset $Y 
\varsupsetneq X$ (maximality):
\end{itemize}}
 \State Apply $X$.
 \If {cache size $k_{ONL}$ is exceeded}:
  \State Empty cache.
  \State Reset $bank_t$ values to be 0.
  \State Start new phase.
 \EndIf
 \EndIf
\EndFor
  \end{algorithmic}
\end{algorithm}

In line 6 of the algorithm above we do not actually exceed the cache. Instead 
we do not apply last valid changeset, empty the cache and start new phase. 
Every phase ends then whenever applying valid changeset $X$ would lead to cache 
being overloaded except of the last phase: this one can be \textit{unfinished}. 
If $P$ is a phase, by $begin_P$ and $end_P$ we denote time when the phase $P$ 
started and finished respectively.

\subsection{Analysis}
Throughout all the analysis we fix the input sequence $I$ and its division into 
phases. We compare \textbf{TRC} with \textbf{OPT} by analyzing both on a single 
phase $P$. By $k_P$ we denote the number of elements in \textbf{TRC's} cache 
after artificial fetch at $end_P$ but before all cache elements eviction.

We start with some basic properties of changesets. Then we introduce 
\textit{fields} which will help us proving algorithm's competitiveness.

\subsubsection{Elementary properties}
Remind that $C_t$ denotes the cache state of \textbf{TRC} at round $t$ and that 
any cache change takes place between rounds in discrete times. 
Having that in mind we prove that at any time there cannot be both saturated 
positive and negative changesets.
\begin{lemma}
\label{thm:lemma1}
 Fix phase $P$. For any $t \in \{begin_P +1, begin_P + 2, \ldots, end_P\}$. 
For any valid changeset $X$ for cache $C_t$ we have $bank_t(X) \leq |X| \cdot 
\alpha$ (1). Moreover changesets $Y$ for $C_t$ for which equality holds are 
all either positive or negative (2).
\end{lemma}
\begin{proof}
We prove this lemma by induction on $t$, starting with $t_1 = begin_P +1$. In 
that time there was only one request so for all $X$ being valid changeset 
$bank_{t_1} \leq 1$, which proves (1). For (2) see that the request can be both 
negative and positive.

To process with the induction step, first we prove following property:
\begin{property}
For any changeset $X$ for $C_{t+1}$ we have $bank_t(X) < |X| \cdot \alpha$.
\end{property}
We prove it only for $X$ being positive changeset. For negative ones the proof 
is symmetric. We consider three cases depending on the action taken by 
algorithm \textbf{TRC} at time $t$.
\begin{enumerate}
 \item \textbf{TRC} does not perform any action. That means that there was no 
saturated (saturation is defined in algorithm \ref{alg:TRC} above) changeset 
$X'$ for $C_t$ and $bank_t(X') \cdot \alpha$ for any $X'$. See that $C_t = 
C_{t+1}$, therefore property follows trivially in that case.
  \item \textbf{TRC} fetches $X'$, so $C_{t+1} = C_t \cup X'$. Assume the 
opposite, that there exists changeset $X$ for $C_{t+1}$ such that $bank_t(X) 
\geq |X| \cdot \alpha$. It implies that $X \cup X'$ is positive changeset for 
$C_t$ that might have been fetched instead of $X'$ at time $t$. That 
contradicts the maximality property of algorithm.
  \item \textbf{TRC} evicts $X'$, so $C_{t+1}  = C_t \setminus X'$. Assume 
again the opposite of the property. It gives, that $X'' = X \ X'$ is valid 
positive changeset for $C_t$ which is saturated, because $bank_t(X'') = 
bank_t(X) \geq |X| \cdot \alpha \geq |X''| \cdot \alpha$. We have both positive 
and negative saturated changeset for $C_t$ which contradicts induction 
assumption (2).
\end{enumerate}
This property implies the lemma for $t+1$ since we have strict inequalities for 
$C_{t+1}$. When request comes in round $t+1$ it is either positive or negative, 
so if it make a changset saturated it can be either positive or negative one. 
Only one request cames during the round, so strict inequalities can only change 
for equalities, so we have first part of lemma, too.
\end{proof}
The above lemma convince us that the algorithm is deterministic when choosing 
changeset to apply.

\subsubsection{Concept of fields}
First notice that whenever there is a positive (negative) request to element 
present (absent)in \textbf(TRC's) cache it neither influence the algorithm 
behaviour nor cost. Therefor from now on we assume that in phase $P$ there are 
no such requests.
We extend the space of requests from nodes to pairs of nodes and time $(v, t)$ 
calling them \textit{slots} and obtained space: \textit{event space}. If there 
is a request to node $v$ at time $t$ we say that slot $(v, t)$ is 
\textit{occupied}. We can then see each phase $P$ as set of occupied slots. 
Figure \ref{fig:spacial_temporal} might help to familiarize with above 
definitions.
\begin{figure}
 \begin{center}
  \includegraphics[width=0.7\textwidth]{spacial_temporal.png}
 \end{center}
 \caption{Event space for node $v$.}
 \label{fig:spacial_temporal}
\end{figure}

As we now know what event space and slots are, we can define $fields$. Fields 
form disjoint partition of phase $P$ (meaning partition of corresponding 
slots). Formally if $X_t$ is changeset applied by \textbf{TRC} at time $t$ we 
define related filed as:
$$ F^t = \{(v, t'): v \in X^t \wedge last_v^t < t' \leq t\},$$
where $last_v^{t'}$ denotes the last time strictly before $t'$ in phase $P$ 
when node $v$ changed its state in cache either by being fetched or evicted. If 
there was no such change for $v$ in phase $P$ we assign $last_v^{t'} = begin_P$.
$F^t$ is \textit{positive} if changeset $X_t$ is positive, symmetrically for 
\textit{negative}. We define $req(F^t)$ to be occupied slots from $F^t$. Notice 
that $req(F^t)$ all exactly the requests that 'payed' for changeset $X_t$. 
Together with Lemma \ref{thm:lemma1} we can summarize it in simple observation.
\begin{observe}
For field $F^t$ we have $req(F^t) = |X_t| \cdot \alpha$ and the requests 
in one field are either all positive or all negative.
\label{obs:observe1}
\end{observe}
The rest of event space, occupied slots that does not belong to any field $F^t$ 
in phase $P$, we call \textit{open field} $F^{\infty}$. All fields except of 
an 
open field are denoted by $\mathcal{F}$ and $|\mathcal{F}| = \sum_{F^t \in 
\mathcal{F}} |X^t| \cdot \alpha$.
Additionally for any field $F$, if $A$ is a set of nodes, $F \cap A = \{(v,t) 
\in F: v \in A$ and if $T$ is set of rounds $F \cap T = \{(v, t) \in F: t \in 
T\}$. To shorten notation we introduce $\tau$\textit{-restricted} subfields of 
$F^t$ as:
$$F^t_{\leq \tau} = F^t \cap {t': t' \leq \tau}.$$
We can think of it as stopping time at $\tau$ and looking what slots are 
occupied up to this time in $F^t$.
Fields are hard to visualize in general case, so we present them on a 
Figure \ref{fig:fields} only for the case when the processed tree structure 
forms a line.
\begin{figure}
\label{fig:fields}
\begin{center}
  \includegraphics[width=1.1\textwidth]{fields.png}
\end{center}
\caption{Example of fields for tree being a line. $F^1$ is a positive field, 
$F^2$ is negative. $F^{\infty}$ is an open field for presented phase.}
\end{figure}

Now we are ready to upper bound the cost of \textbf{TRC} on phase $P$.
\begin{lemma}
Let $\mathcal{F} \cup F^{\infty}$ be a partition of $P$ into fields. We have:
$$TRC(P) \leq 2 \cdot \alpha \cdot |\mathcal{F}| + req(F^{\infty}) + k_P \cdot 
\alpha.$$ 
\end{lemma}
\begin{proof}
The cost on every field $F^t$ originates from serving requests belonging to it 
and to reorganize the cache at the end of the field - both this costs are equal 
$|X^t| \cdot \alpha$, which sums up to the first term in right side of 
inequality. Expression $req(F^{\infty})$ captures all the cost associated with 
open field in $P$ and $k_P$ is the cost of final eviction which does not 
occur for the unfinished (last) phase. 
\end{proof}

\subsubsection{Shifting}
Now we process to the most tricky part of our deduction. What we need to do is 
relate the cost of \textbf{OPT} to fields. Just to show the motivation, imagine 
for a moment that requests are evenly distributed within fields. What we mean, 
is every node contained in every field $F^t$ has exactly $\alpha$ slots 
saturated at time $t$ (so we distribute $|X^t| \cdot \alpha$ requests belonging 
to $F^t$ evenly). What it gives us is that every node $v$ alternates between 
periods spent in positive and negative field and on each such periods it is 
requested $\alpha$ times. Moreover, we see that on two consecutive periods 
(one positive and one negative) \textbf{OPT} pays at least $\alpha$ changing 
the state of $v$ or paying for all positive or all negative requests. 
Unfortunately, we do not have such fair distribution of requests. What we do is 
we change phase $P$ to $P'$ which is not harder for \textbf{OPT} but will be 
close enough to even distribution to lower bound the cost.

To achieve that we will change $P$ by applying \textit{legal shifts}. The 
requests placed in slot $(v, t)$ can be legally shifted to slot $(m(v), t)$ if 
$v = m(v)$ or for positive requests $m(v)$ is descendant of $v$, for negative 
$m(v)$ is ascendant of $v$. See, that for fixed sequence of fetches and 
removals, for any algorithm, the associated cost can only decrease when this 
legal shifts are applied. For example for positive request, shifting it 'down' 
in the tree can lead to a node which is already in cache and we can save on 
that request. Similar example can be constructed for negative requests. So we 
can make following observation:
\begin{observe}
 If $P'$ was created from $P$ by application of  legal shifts then 
$cost_{OPT}(P') \leq cost_{OPT}(P)$.
\end{observe}

We do not want $P'$ to lead \textbf{OPT} to have too small cost so we cannot 
argue about it is sufficiently large. What we want to keep in our hands is that 
there are still consecutive alternating periods of positive and negative 
requests that are sufficiently 'long' in the sense of number of requests. 
Therefore we will show that we can perform shifts within fields in such way 
that there are exactly (for negative requests) or approximately (for positive 
requests) distributed, so we can lower bound \textbf{OPT}.

Before we start shifting the requests we present a property of restricted 
subfields and corresponding changeset. For any filed $F$ we say that $S 
\subseteq F$ is \textit{overrequested} if $req(S) > |S| \cdot \alpha$. It turns 
out that some subsets of fields are never overrequested.

\begin{lemma}
Fix $F^t$, $F^t_{\leq \tau}$ with $X^t$ corresponding changeset.
\begin{itemize}
 \item If $F^t$ is negative then for any tree cap $T'$ of $X^t$, $T' \cap 
F^t_{\leq \tau}$ is not overrequested.
\item If $F^t$ is positive then for any subtree $T'\cap F^t_{\leq \tau}$ is not 
overrequested.
\end{itemize}
\label{thm:not_over_requested}
\end{lemma}
\begin{proof}
We prove the property for negative fields only. The proof for positive ones is 
similar. From \ref{thm:lemma1}, because $T' \cap  F^t_{\leq \tau}$ is valid 
changeset at $\tau$, $req(T' \cap  F^t_{\leq \tau}) = bank_{\tau}(T' \cap  
F^t_{\leq \tau}) \leq |D \cap F^t_{\leq \tau}| \cdot \alpha$.
\end{proof}
\begin{wrapfigure}{l}{0.5\textwidth}
\vspace{-20pt}
 \begin{center}
  \includegraphics[width=0.5\textwidth]{density_pic.png}
\end{center}
\caption{This picture shows the average density of request in fields. The 
darker colour the more request is in average in that part of changeset.}
\vspace{-40pt}
\label{fig:density}
\end{wrapfigure}
For $\tau = t$ we can apply this lemma and Observation \ref{obs:observe1} 
what gives us a property of fields, intuitively saying that for positive 
fields the saturation of slots in average is more dense the closer we are to 
the root, whereas the density of requests in negative field is higher in the 
bottom (see Figure \ref{fig:density}).
\begin{corollary}
 Fix $F^t$ filed with corresponding changeset $X_t$. Let $T'$ be tree cap of 
$X_t$, then if the field is:
 \begin{itemize}
  \item positive: $req(F^t \cap T') \geq |T'| \cdot \alpha$,
  \item negative: $req(F^t \cap (X_t \setminus T')) \geq |X_t \setminus 
T'| \cdot \alpha$.
 \end{itemize}
 \label{thm:proper_at_t}
\end{corollary}

\myparagraph{Up-shifting of negative requests}
With up shifting negative requests up, we can obtain an even distribution of 
requests across nodes in changeset.
\begin{theorem}
By legal shifts of negative requests we can move request within the field $F^t$ 
and each node is requested exactly $\alpha$ times.
\label{thm:legal_shifting_up}
\end{theorem}
To prove this theorem we will first convince ourselves that we can always 
shift requests that appeared after first $\alpha$ requests to leaf to its 
parent. To make our deduction transparent we will say that tree cap $Y 
\subseteq X_t$ is \textit{proper} if for any tree cap $Y' \subseteq Y$ and any 
time $\tau \leq t$ it holds that $req(F^t_{\leq \tau} \cap Y') \leq |Y'| \cdot 
\alpha$. Look that Corollary \ref{thm:proper_at_t} tells that at time $t$, 
$X_t$ with all its tree caps are proper.

Now we can present a lemma which will be then used as induction step to prove 
Theorem \ref{thm:legal_shifting_up}.
\begin{lemma}
 Let $Y$ be a proper tree cap of negative changeset $X_t$ and let $v_l$ be one 
of $Y$'s leaves. If $v_l$ was requested $r_l \geq \alpha$ we can legally shift  
$\delta_l = r_l - \alpha$ requests to parent of $v_l$ ($p(v_l)$), so they are 
kept in $F^t$ and $Y \setminus \{v_l\}$ stays proper. 
\end{lemma}
\begin{proof}
We consider $r_l$ requests that are placed on $v_l$ in chronological order. See 
that when $\alpha + 1$-th request comes, at time $t_l$, $p(v_l)$ has to be in 
cache. If that was not true, we would have overrequested tree cap 
${v_l}$ which 
contradicts Lemma \ref{thm:not_over_requested}. That means, after $\alpha$ first 
requests to $v_l$ the rest $\delta_l$ requests can be moved within field to 
$p(v_l)$.

What remains to be proven is that $Y' = Y \setminus {v_l}$ after that operation 
is proper. Suppose it is not, and there exists tree cap $D \subseteq Y'$ which 
is overrequested in some time $\tau$. First of all $\tau \geq t_l$ because we 
do not change placement of any request up to time $t_l$ and $Y$ before change 
was not overrequested. Also see that $p(v) \in D$ - we did not change the 
number of requests in any other node. After shift $req(F^t \cap Y) = req(F^t 
\cap D) + \alpha$, and if $req(F^t \cap D) > |D| \cdot \alpha$, then $D \cup 
v_l$ would be overrequested contradicting that $Y$ is proper before change.
\end{proof}
\begin{wrapfigure}{r}{0.5\textwidth}
\vspace{-20pt}
 \begin{center}
  \includegraphics[width=0.5\textwidth]{example_not_even.png}
\end{center}
\caption{Numbers indicate chronology of events.}
\vspace{-20pt}
\label{fig:example_not_even}
\end{wrapfigure}
Now above lemma, used as induction step, proves Theorem 
\ref{thm:legal_shifting_up}. To see that remind that 
Corollary \ref{thm:proper_at_t} ensures that in any proper tree cap $D$ of 
$X_t$ there is a leaf with at least $\alpha$ requests. Going with bottom-up 
manner through the field shifting requests from 'overrequested' the proof of 
theorem is completed.   
\myparagraph{Down-shifting of positive requests}
Unfortunately, we can not obtain such even distribution of requests by legal 
shifts in a positive field as in the negative case. Specifically, we cannot 
shift legally positive requests down and end up with $\alpha$ requests in every 
node. Imagine we have a tree $T$ with root $r$ and two distinct subtrees $T_1$ 
and $T_2$ both of size $s$ and having $l$ leaves. Consider following series on 
events, assuming that initially whole tree $T$ is in \textbf{TRC's} cache.
\begin{enumerate}
 \item $T_1 \cup \{r\}$ is evicted from cache.
 \item $(s+1) \cdot \alpha - l$ requests to root $r$ appear (no fetch occurs - 
there is not enough requests in $r$ to pay for whole $T_1 \cup \{r\}$).
 \item $T_2$ is evicted.
 \item $s \cdot \alpha$ requests appear at root of $T_1$.
 \item $l$ requests appear at root $r$ of tree $T$. $T$ is again in 
\textbf{TRC's} cache.
\end{enumerate}
This example's picture can be found on Figure \ref{fig:example_not_even}. 
Notice that when first $(s+1) \cdot \alpha - l$ request to $r$ appear they can 
only be shifted to $T_1$ nodes. After that root of $T_1$ is given requests 
which can not be moved to $T_2$ - we are unable to do shift up with positive 
requests. Therefore only $l$ last requests are left which can be moved to 
$T_2$. Assuming that $\frac{l}{\alpha} \ll s$ we only managed to fill half of 
the tree nodes with $\Omega(\alpha)$ requests.   

Convinced that shifting down request is not so easy, our goal is to prove that 
we can get, by legal shifting within field $F^t$, 
at least $\Omega(\frac{|F^t|}{h(T)})$ nodes having at least $\frac{\alpha}{2}$ 
requests. To do that first we argue that from any node $v$ in the positive 
field $F^t$ we can move down fraction of its requests that we stay in the field 
and no node $u$ from $T(v)$ gets more then $\frac{\alpha}{2}$ (we do not count 
the requests placed originally at $u$). Notice that to shift any request $(v, 
t_v)$ from $F^t$ to node $v'$ we need to ensure that $v$ is outside the cache, 
what can be expressed by $t_v > last_u(t)$.

\begin{lemma}
Fix $F^t$ positive field with corresponding $X_t$ and $v \in X_t$ that has 
been requested at least $c_v \cdot (\frac{\alpha}{2})$ times for some positive 
natural number $c_v$. There exists legal shifting of $\lceil \frac{\alpha}{2} 
\rceil \cdot (\frac{\alpha}{2})$ requests from the requests to $v$ to nodes 
$T(v) \cap X^t$ and at least $\lceil \frac{\alpha}{2} \rceil$ of them is given 
$\frac{\alpha}{2}$. All these shifts are within the field $F^t$.
\label{thm:request_mapping}
\end{lemma}
\begin{proof}
We sort the elements of $T(v) \cap X_t$ with respect to increasing time of last 
eviction (function $last_v$), getting sequence of nodes $\{u_i\}_{i \geq 
0}$. Since $v$ has to be on of the first elements that are evicted from cache 
in that positive field, we assume $u_0 = v$. We sort the request sequence to 
$v$ too to get the sequence with increasing arrival time. The resulting 
sequence will be denoted by $\{(v, t_j)\}_{j=0}^{c_v \cdot \frac{\alpha}{2} - 
1}$. We define partial function \textit{sh} on latter sequence's 
elements, that will say to which element of $T(v) \cap X_t$ request should be 
pushed down. Specifically, $sh((v, t_{k \cdot \alpha + m})) = u_k$, where $m$ 
values are from $[0, \frac{\alpha}{2})$. The function $sh$ is not defined on 
the rest of requests, which mean we will not push those requests down. The 
mapping definition is presented on Figure \ref{fig:req_map}.
\begin{figure}
 \begin{center}
  \includegraphics[width=1.1\textwidth]{request_mapping.png}
 \end{center}
 \caption{Request mapping from Lemma \ref{thm:request_mapping}. The requests 
are divided into alternating parts of size $\lceil \frac{\alpha}{2} \rceil$ 
and $\lfloor \frac{\alpha}{2} \rfloor$. The odd fragments (counting from $1$) 
are mapped, whereas even parts are omitted in mapping}
 \label{fig:req_map}
\end{figure}

Now we need to prove that defined shifting is within the field, meaning that 
for shifted requests $t_j \geq last_{sh((v, t_j))}(t)$. Assume that it is not 
true, so $t_j < last_{sh((v, t_j))}(t)$ for $j = k \cdot \alpha + m$ 
corresponding to shifted request. Since request are mapped with respect to 
increasing order of sequence $\{u\}$, $|F^t_{\leq t_j}| \cap T(v)| <= k|$, but 
the same time $req(F^t_{\leq t_j}| \cap T(v)) = j$ and $F^t_{\leq t_j}| \cap 
T(v)$ is then overrequested. This with Lemma \ref{thm:not_over_requested} leads 
us to contradiction.
\end{proof}
We are ready to prove the main lemma of this paragraph, which gives as 
possibility of obtaining even distribution of positive request to the factor 
$O(\frac{1}{h(T)})$.
\begin{lemma}
For any positive field $F^t$ it is possible to shift 
legally $\frac{req(F^t)}{2}$ requests within the field, so at least 
$\frac{|X^t|}{2 \cdot h(T)}$ nodes in $F^t$ have at least 
$\frac{\alpha}{2}$ requests each.
\label{thm:legal_shifting_down}
\end{lemma}
 \begin{figure}
 \begin{center}
  \includegraphics[width=0.5\textwidth]{layers.png}
\end{center}
\caption{Division of requests to layers. Requests to nodes which have the same 
distance from $X^t$ root are on the same level.}
\label{fig:layers}
\end{figure}
\begin{proof}
Let $X^t$ be corresponding changeset for $F^t$. We take all the requests in 
every node and arrange them into groups of size $\frac{\alpha}{2}$ 
chronologically, meaning first $\frac{\alpha}{2}$ requests to node are in first 
group and so on. If the last group is not full (it has less then  
$\frac{\alpha}{2}$ requests) then it is omitted. The number of request that are 
not omitted in set $X$ will be denoted by $\overline{req}(X)$. Notice that  
$\overline{req}(F^t) \geq |X^t| \cdot \frac{\alpha}{2}$, since $req(F^t) = 
|X^t| \cdot \alpha$ and we omit at most $\frac{\alpha}{2}$ requests at every 
node.

We partition the elements of $X^t$. In each part there are elements with the 
same distance from root of $X^t$ and call them levels. The nodes that are 
distance $i$ from root are on level $i$ and are denoted $X^t_i$. We therefore 
have exactly $h(T)$ layers. By pigeonhole principle there exists a level 
$X^t_i$ for which $\overline{req}(F^t \cap X^t_i) \geq \frac{req(F^t)}{2} \cdot 
h(T)$ (look at Figure \ref{fig:layers}). The subtrees rooted in nodes from 
$X^t_i$ are distinct and to each of them we apply Lemma 
\ref{thm:request_mapping}. Node $v \in X^t_i$ has $\overline{req}(F^t \cap 
\{v\})$ requests so from the lemma we we can shift $\frac{\alpha}{2}$ of 
them to $\lceil \overline{req}(F^t \cap \{v\} \rceil$ distinct nodes in $T(v)$. 
Summing up the 'filled' nodes for each element from $X^t_i$ we get at least 
$\lceil \overline{req}(F^t \cap X^t_i) \rceil \geq \frac{|X^t|}{2 \cdot h(T)}$ 
with $\frac{\alpha}{2}$ requests.
\end{proof}
\subsubsection{Competitiveness of TRC}
We will use results from previous subsection to lower bound \textbf{OPT's} cost 
and finally getting the main theorem of the thesis about competitive ratio of 
\textbf{TRC}. This proofs are mostly technical, but still intuitions from 
previous sections might be useful to understand idea behind them.

We will need one auxiliary lemma about \textbf{TRC's} cache at the end of the 
phase. We remind that this last cache state is special - its size exceeds 
$k_{ONL}$, because instead of normal fetch we perform artificial one 
described earlier in this section.  
\begin{lemma}
Fix phase $P$ and $C$ cache state at the end of the phase ($end_P$). Any tree 
cap $D \subseteq C$ received at least $|D| \cdot \alpha$ requests during phase 
$P$.
\label{thm:lots_of_req_in_tc_end_of_p}
\end{lemma}
\begin{proof}
Let $X_1, X_2, \ldots, X_m$ be all positive chengesets from phase $P$. We make 
separation by defining $X'_i = X_i \setminus \bigcup_{j=1}^{i-1} X_j$. Note 
that $X'_i$ is tree cap of $X_i$. The sequence $X'_1, X'_2, \ldots, X'_m$ is 
pairwise disjoint and covers $C$, meaning that $C \subseteq \bigsqcup_{j=1}^m 
X'_j$. Let $X_j^D = D \cap X'_j$, then $D = \bigsqcup_{j=1}^m X_j^D$ since $D 
\subseteq C$. $X_j^D$ is tree cup of $X'_j$ so by transitivity of 
being a tree cap relation, $X_j^D$ makes up a tree cap of $X_j$. From Corollary 
\ref{thm:proper_at_t} we have that $req(X_j^d \cap F_i) \geq |X_j^D| \cdot 
\alpha$ so it can be even more on the whole phase - not only restricted to 
field $F_i$. Since $X_j^D$ are pairwise disjoint, we can sum up requests over 
all $j$ what proves the lemma.
\end{proof}

\myparagraph{Lower bound for OPT}
In this paragraph we use a notation of $V_{OPT}$ - it will be the set of nodes 
that appeared in \textbf{OPT's} cache during phase and $V_{OPT}^C$ will 
denote the complement to the whole tree $T$ ($T \setminus V_{OPT}$). Notice, 
that $V_{OPT}$ creates a proper cache state.

We get two different lower bounds for \textbf{OPT} presented in the next two 
lemmas. 
\begin{lemma}
For any phase $P$, $cost_{OPT}(P) \geq (k_P - k_{OPT}) \cdot \alpha$. 
\end{lemma}
\begin{proof}
Let $C$ be the cache state of \textbf{TRC} at $end_P$ (after artificial fetch, 
so it has $k_P$ elements in cache). We consider separately the cost 
\textbf{OPT} obtained on the nodes $V_{OPT}$ and $V_{OPT}^c$. For the first 
set, \textbf{OPT} had to fetch at least $|V_{OPT}| - k_{OPT}$ nodes during the 
phase paying $\alpha$ for each of them. For the second one, see that 
$C \setminus V_{OPT}$ is a subset of $V_{OPT}^c$. Moreover, $C \setminus 
V_{OPT}$ is an union of disjoint tree caps of subtrees belonging to $C$. From 
Lemma \ref{thm:lots_of_req_in_tc_end_of_p} all those tree caps are dense, so 
summing requests in them gives us at least $|C \setminus V_{OPT}| \cdot 
\alpha$ requests to nodes in $C \setminus V_{OPT}$ which belongs to $V_{OPT}^c$ 
as mentioned above. Joining results for both cases, $cost_{OPT}(P) \geq 
(|V_{OPT}| - k_{OPT}) \cdot \alpha + |C \setminus V_{OPT}| \cdot 
\alpha = (|C| - k_{OPT}) \cdot \alpha$ what ends the prove since $k_P$ is 
defined as $|C|$ after artificial fetch. 
\end{proof}
\begin{lemma}
For any phase $P$, $cost_{OPT}(P) \geq (\frac{|\mathcal{F}|}{4 \cdot h(T)} 
-k_P) \cdot \frac{\alpha}{2}$. 
\end{lemma}
\begin{proof}
Let $P$ be a fixed phase. We perform shifting on $P$ so that Lemma 
\ref{thm:legal_shifting_up} and Lemma \label{thm:legal_shifting_down} hold. The 
result of this action will be called phase $P'$. From the previous discussion 
it is sufficient to prove that $cost_{OPT}(P') \geq (\frac{|\mathcal{F}|}{4 
\cdot h(T)}-k_P) \cdot \frac{\alpha}{2}$. To do that we first focus on each $v$ 
node separately. See that it's history can be divided into alternating 
intervals when it was out of the cache and was charged by receiving the positive 
requests or it was in the cache charged by negative requests. We call those 
periods $OUT$ and $IN$ respectively. Cache is empty in the beginning of the 
phase, so every node starts with $OUT$ period. Two consecutive $OUT$ and $IN$ 
periods form \textit{stripe}. Notice, that for every node $v$ only the last 
$OUT$ period in phase $P'$ might be unpaired, therefore there is at most $k_P$ 
unpaired periods.

Let's denote by $cnt_{IN}$ and $cnt_{OUT}$ the number of all $IN$ and $OUT$ 
periods in $P$ respectively. From the above discussion, $cnt_{OUT} = cnt_{IN} + 
k_P$. The sum of all periods (both $IN$ and $OUT$) is equal to the sum of sizes 
of all fields (excluding the open one) which we denoted by $|\mathcal{F}|$. It 
implies that $|\mathcal{F}| = cnt_{OUT} + cnt_{IN}$ so $cnt_{OUT} = 
\frac{|\mathcal{F}| + k_P}{2} \geq \frac{|\mathcal{F}|}{2}.$

Now we will use the lemmas about shifting. From them we know that in $P'$ all 
$IN$ periods consist of $\alpha$ requests and there is at least 
$\frac{1}{2h(T)}$ throughout $OUT$ periods paired with $IN$ periods that have 
at least $\frac{\alpha}{2}$ requests. Therefore there are at least 
$(\frac{cnt_{OUT}}{2h(T)} - k_P)$ pairs of $OUT$ and $IN$ periods that both 
have $\frac{\alpha}{2}$ requests or more. On each such pair \textbf{OPT} pays 
at least $\frac{\alpha}{2}$ because of changing nodes state in cache (at cost 
$\alpha$) or paying from either positive or negative requests from that stripe. 
Stripes are distinct, so the cost of \textbf{OPT} can be bounded from below by 
$(\frac{cnt_{OUT}}{2h(T)} - k_P) \cdot \frac{\alpha}{2} \geq 
(\frac{|\mathcal{F}|}{4h(T)} - k_P) \cdot  \frac{\alpha}{2}$.
\end{proof}

\myparagraph{Upper bound for open fields}
\myparagraph{Main result}
\myparagraph{Lower bound for competitive ratio}
