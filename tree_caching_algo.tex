\section{Tree caching with dependencies}
The problem we consider (called \textit{tree caching problem})is an extension 
of classic caching problem with bypassing that was recalled in secion 2. The 
main difference is that requested items have inter-dependencies. Specifically, 
universe forms rooted tree $T$ and the items requested are placed in its nodes 
and whenever $v \in T$ is fetched all of its descendants $T(v)$ that are not in 
cache have to be fetched at the same time. We say that the cache with property 
described above is \textit{bottom-contiguous}.

Moreover, we consider two types of requests: \textit{negative} and 
\textit{positive}. If incomming request is positive we pay 1 if it is 
not present in the cache (so the same as in most considered settings). On the 
other hand we pay for negative requests if the item is cached (so in opposite 
situation). After serving of any request we can reorgenize cache paying 
$\alpha \geq 1$ for every element evicted or fetched. 

The goal is to minimize the cost obtained by online algorithm solving this 
problem. We present a deterministic algorithm \textbf{TRC} which we prove is 
$O(h(T) \times R)$-competitive where $h(T)$ is height of input tree and $R = 
k_{ONL}/(k_{ONL} - k_{OPT} +1$ with $k_{ONL}$ and $k_{OPT}$ being size of 
online and optimal offline algorithm cache.

\subsection{Model}
\begin{wrapfigure}{r}{0.6\textwidth}
\begin{postscript}

\psscalebox{0.6 0.6} % Change this value to rescale the drawing.
{
\begin{pspicture}(0,-4.0099998)(12.562361,4.0099998)
\definecolor{colour0}{rgb}{0.78431374,0.78431374,0.78431374}
\psbezier[linecolor=black, linewidth=0.04, 
fillstyle=solid,fillcolor=colour0](4.8323607,3.9899998)(3.8323605,
3.9899998)(0.046042196,0.2078214)(0.8323605,-0.41000015)(1.6186789,
-1.0278217)(9.759687,-1.1499403)(10.432361,-0.41000015)(11.105033,
0.32993993)(5.8323607,3.9899998)(4.8323607,3.9899998)
\pscircle[linecolor=black, linewidth=0.04, fillstyle=solid,fillcolor=black, 
dimen=outer](4.8323607,2.79){0.4}
\pscircle[linecolor=black, linewidth=0.04, fillstyle=solid,fillcolor=black, 
dimen=outer](3.2323606,1.9899999){0.4}
\pscircle[linecolor=black, linewidth=0.04, fillstyle=solid,fillcolor=black, 
dimen=outer](4.8323607,1.1899998){0.4}
\pscircle[linecolor=black, linewidth=0.04, fillstyle=solid,fillcolor=black, 
dimen=outer](6.4323606,1.9899999){0.4}
\pscircle[linecolor=black, linewidth=0.04, fillstyle=solid,fillcolor=black, 
dimen=outer](8.03236,1.1899998){0.4}
\pscircle[linecolor=black, linewidth=0.04, fillstyle=solid,fillcolor=black, 
dimen=outer](7.6323605,-0.41000015){0.4}
\pscircle[linecolor=black, linewidth=0.04, fillstyle=solid,fillcolor=black, 
dimen=outer](9.232361,-0.0100001525){0.4}
\psline[linecolor=black, linewidth=0.04](4.8323607,2.79)(3.2323606,1.9899999)
\psline[linecolor=black, linewidth=0.04](4.8323607,2.79)(4.8323607,1.1899998)
\psline[linecolor=black, 
linewidth=0.04](4.8323607,2.79)(6.4323606,1.9899999)(6.4323606,1.9899999)
\psline[linecolor=black, 
linewidth=0.04](6.4323606,1.9899999)(8.03236,1.1899998)(7.6323605,
-0.41000015)(8.03236,1.1899998)(9.232361,-0.0100001525)(9.232361,-0.41000015)
\psline[linecolor=black, 
linewidth=0.04](3.2323606,1.9899999)(3.2323606,1.9899999)
\psline[linecolor=black, 
linewidth=0.04](4.8323607,2.79)(4.8323607,2.79)(4.8323607,2.79)
\pstriangle[linecolor=black, linewidth=0.06, 
dimen=outer](1.2323605,-3.6100001)(2.4,2.8)
\pstriangle[linecolor=black, linewidth=0.06, 
dimen=outer](4.8323607,-3.6100001)(2.4,2.8)
\psline[linecolor=black, 
linewidth=0.04](4.8323607,0.78999984)(4.8323607,-0.8100002)(4.8323607,
-0.8100002)
\pstriangle[linecolor=black, linewidth=0.06, 
dimen=outer](9.232361,-3.6100001)(2.4,2.8)
\psline[linecolor=black, 
linewidth=0.04](2.8323605,-0.0100001525)(2.8323605,-0.0100001525)(2.8323605,
-0.0100001525)(2.8323605,-0.0100001525)
\rput(11.63236,1.9899999){Tree cap}
\rput(9.232361,-0.0100001525){\textcolor{white}{\textbf{r}}}
\pspolygon[linecolor=black, linewidth=0.04, fillstyle=vlines, 
hatchwidth=0.028222222, hatchangle=0.0, 
hatchsep=0.1411111](2.4323606,-3.6100001)(1.6323606,-2.41)(1.2323605,
-2.8100002)(0.8323605,-2.0100002)(0.032360535,-3.6100001)(0.032360535,
-3.6100001)
\pspolygon[linecolor=black, linewidth=0.04, fillstyle=vlines, 
hatchwidth=0.028222222, hatchangle=0.0, 
hatchsep=0.1411111](4.4323606,-3.6100001)(4.8323607,-2.41)(6.0323606,
-3.6100001)(6.0323606,-3.6100001)
\psframe[linecolor=black, linewidth=0.04, fillstyle=vlines, 
hatchwidth=0.028222222, hatchangle=0.0, hatchsep=0.1411111, 
dimen=outer](10.432361,3.1899998)(9.63236,2.3899999)
\rput(11.63236,2.79){Valid cache }
\psline[linecolor=black, 
linewidth=0.04](9.232361,-1.6100001)(9.232361,-1.6100001)(9.232361,
-1.6100001)(9.232361,-1.6100001)
\psline[linecolor=black, 
linewidth=0.04](9.232361,-0.0100001525)(9.232361,-0.8100002)(9.232361,
-0.8100002)
\pstriangle[linecolor=black, linewidth=0.06, linestyle=dotted, 
dotsep=0.10583334cm, fillstyle=vlines, hatchwidth=0.02, hatchangle=0.0, 
hatchsep=0.2212, dimen=outer](9.232361,-4.01)(4.0,5.2)
\psline[linecolor=black, 
linewidth=0.04](3.2323606,1.9899999)(1.2323605,-0.8100002)(1.2323605,-0.8100002)
\psline[linecolor=black, 
linewidth=0.04](3.2323606,1.1899998)(3.2323606,1.1899998)
\psline[linecolor=black, 
linewidth=0.04](4.8323607,3.59)(4.8323607,3.59)(4.8323607,3.59)(4.8323607,3.59)
\psframe[linecolor=black, linewidth=0.04, fillstyle=solid,fillcolor=colour0, 
dimen=outer](10.432361,2.3899999)(9.63236,1.5899998)
\psframe[linecolor=black, linewidth=0.064, linestyle=dotted, 
dotsep=0.10583334cm, fillstyle=solid, 
dimen=outer](10.432361,3.9899998)(9.63236,3.1899998)
\rput(11.63236,3.59){T(r)}
\end{pspicture}
}

\end{postscript}
\caption{Visualization of definitions.}
\label{fig:TreeCacheDefinitions}
\end{wrapfigure}
For ease of future algorithm description and readability of proofs we introduce 
some usefull definitions. First of all by $T(r)$ we mean subtree of $T$ rooted 
in $r$ and by $h(T)$ height of tree. \textit{Tree cap} is a tree $T_c \subseteq 
T$ such that root $r$ of $T$ belongs to it and $v \in T_c$ implies that all the 
nodes on the path from $v$ to $r$ belong to $T_c$. We say that set $C \subseteq 
T$ is \textit{bottom contiguous} if $v \in C$ implies $T(v) \subseteq C$ and we 
say that set $C$ \textit{valid cache state}. Notice that it is not dependent on 
cache size so $C$ ca be valid cache state and not fit into cache size the same 
time.

The \textbf{TRC} algorihm, after each request, can change the state of cache, 
but the only possible changes have to leave the the cache in valid state. 
Depending on the character of change (eviction vs. fetch) we can apply for cahce 
$C$ either \textit{valid negative changeset} or \textit{valid positive 
changeset}. The valid positive changeset $X$ (think of it as a set that can be 
fetched) is a set for which $X \cap C = \varnothing$ and $C \cup X$ 
is a valid cache state, the negative one (related to evicion) is a set $X 
\subseteq C$ such that $C \ X$ is valid cache state.

We process the input one by one and assume each request is processed in 
discrete time $t$, so the next request is served in time $t + 1$ and so on. In 
the time interval an changes to cache take place, thouse $C_t$, cache state at 
time $t$ is well defined.

Some of the definition which appreared in this section are presented on figure 
~\ref{fig:TreeCacheDefinitions}.

\subsection{Algorithm TRC}

Before finally defining the \textbf{TRC} algorithm we bind to each node $v \in 
T$ a $bank_{t}(v)$ value. At the beginning of algorithm for each $v \in T$ we 
have $bank_{0}(v) = 0$. If node $v$ gets negative or positive request at 
time $t$ for which we pay 1 we increase $bank_{t}(v) = bank_{t-1}(v) + 1$, 
leaving the value for any other node unchanged (so for $v' \neq v 
bank_{t-1}(v') = bank_{t}(v')$). When we change the state of node $v$ (either 
evict or fetch it into cache) we clear its bank value. For set of nodes $V 
\subseteq T$, we denote sum $\sum_{v \in V} bank_{t}(v)$ as $bank_{t}(V)$.

The \textbf{TRC} proceedes in phases. First phase starts at time 0 and at the 
start of each phase the \textbf{TRC} cache is empty. 
\begin{algorithm}
\caption{\textbf{TRC}}
\begin{algorithmic}[1]
\ForAll{$t \in \{1 \ldots |\sigma|\}$}
  \State Serve request $\sigma(t)$.
  \State Compute $bank_t(v)$ values for $v \in T$.
  \If{exists valid changeset $X$ such that: \begin{itemize}
    \item $bank_t(X) \geq |X| \times \alpha$ (saturation),
    \item $bank_t(Y) < |Y| \times \alpha$ for any valid changeset $Y 
\varsupsetneq X$ (maximality):
\end{itemize}}
 \State Apply $X$.
 \If {cache size $k_{ONL}$ is exceeded}:
  \State Empty cache.
  \State Reset $bank_t$ values to be 0.
  \State Start new phase.
 \EndIf
 \EndIf
\EndFor
  \end{algorithmic}
\end{algorithm}

In line 6 of the algortihm above we do not actually exceed the cache. Instead 
we do not apply last valid changeset, empty the cache and start new phase. 
Every phase ends then whenever applying valid changeset $X$ would lead to cahce 
being overloaded except of the last phase: this one can be \textit{unfinished}.