\section{Tree caching with dependencies}
\label{tree_caching_algo}
The problem we consider (called \textit{tree caching problem}) is an extension 
of standard caching problem with bypassing that was recalled in section 2. The 
main difference is that requested items have inter-dependencies. Specifically, 
universe forms rooted tree $T$ and the items requested are placed in its nodes 
and whenever $v \in T$ is fetched all of its descendants $T(v)$, that are not 
in 
cache, have to be fetched at the same time. We say that the cache with property 
described above is \textit{bottom-contiguous}.

Moreover, we consider two types of requests: \textit{negative} and 
\textit{positive}. If incoming request is positive we pay 1 if it is 
not present in the cache (so the same as in most considered settings). On the 
other hand, we have to pay for negative requests if and only if the item is 
cached (so in opposite 
situation to the positive case). After serving request we can reorganize 
cache paying 
$\alpha \geq 1$ for every element evicted or fetched. 

The goal is to minimize the cost obtained by online algorithm solving this 
problem. We present a deterministic algorithm \textbf{TRC}, which we prove is 
$O(h(T) \times R)$-competitive where $h(T)$ is height of input tree and $R = 
k_{\mathrm{ONL}}/(k_{\mathrm{ONL}} - k_{\mathrm{OPT}} +1$ with 
$k_{\mathrm{ONL}}$ and $k_{\mathrm{OPT}}$ being size of 
online and optimal offline algorithm cache.

\subsection{Model}
\begin{figure}
\begin{center}
\begin{postscript}

\psscalebox{0.6 0.6} % Change this value to rescale the drawing.
{
\begin{pspicture}(0,-4.0099998)(12.562361,4.0099998)
\definecolor{colour0}{rgb}{0.78431374,0.78431374,0.78431374}
\psbezier[linecolor=black, linewidth=0.04, 
fillstyle=solid,fillcolor=colour0](4.8323607,3.9899998)(3.8323605,
3.9899998)(0.046042196,0.2078214)(0.8323605,-0.41000015)(1.6186789,
-1.0278217)(9.759687,-1.1499403)(10.432361,-0.41000015)(11.105033,
0.32993993)(5.8323607,3.9899998)(4.8323607,3.9899998)
\pscircle[linecolor=black, linewidth=0.04, fillstyle=solid,fillcolor=black, 
dimen=outer](4.8323607,2.79){0.4}
\pscircle[linecolor=black, linewidth=0.04, fillstyle=solid,fillcolor=black, 
dimen=outer](3.2323606,1.9899999){0.4}
\pscircle[linecolor=black, linewidth=0.04, fillstyle=solid,fillcolor=black, 
dimen=outer](4.8323607,1.1899998){0.4}
\pscircle[linecolor=black, linewidth=0.04, fillstyle=solid,fillcolor=black, 
dimen=outer](6.4323606,1.9899999){0.4}
\pscircle[linecolor=black, linewidth=0.04, fillstyle=solid,fillcolor=black, 
dimen=outer](8.03236,1.1899998){0.4}
\pscircle[linecolor=black, linewidth=0.04, fillstyle=solid,fillcolor=black, 
dimen=outer](7.6323605,-0.41000015){0.4}
\pscircle[linecolor=black, linewidth=0.04, fillstyle=solid,fillcolor=black, 
dimen=outer](9.232361,-0.0100001525){0.4}
\psline[linecolor=black, linewidth=0.04](4.8323607,2.79)(3.2323606,1.9899999)
\psline[linecolor=black, linewidth=0.04](4.8323607,2.79)(4.8323607,1.1899998)
\psline[linecolor=black, 
linewidth=0.04](4.8323607,2.79)(6.4323606,1.9899999)(6.4323606,1.9899999)
\psline[linecolor=black, 
linewidth=0.04](6.4323606,1.9899999)(8.03236,1.1899998)(7.6323605,
-0.41000015)(8.03236,1.1899998)(9.232361,-0.0100001525)(9.232361,-0.41000015)
\psline[linecolor=black, 
linewidth=0.04](3.2323606,1.9899999)(3.2323606,1.9899999)
\psline[linecolor=black, 
linewidth=0.04](4.8323607,2.79)(4.8323607,2.79)(4.8323607,2.79)
\pstriangle[linecolor=black, linewidth=0.06, 
dimen=outer](1.2323605,-3.6100001)(2.4,2.8)
\pstriangle[linecolor=black, linewidth=0.06, 
dimen=outer](4.8323607,-3.6100001)(2.4,2.8)
\psline[linecolor=black, 
linewidth=0.04](4.8323607,0.78999984)(4.8323607,-0.8100002)(4.8323607,
-0.8100002)
\pstriangle[linecolor=black, linewidth=0.06, 
dimen=outer](9.232361,-3.6100001)(2.4,2.8)
\psline[linecolor=black, 
linewidth=0.04](2.8323605,-0.0100001525)(2.8323605,-0.0100001525)(2.8323605,
-0.0100001525)(2.8323605,-0.0100001525)
\rput(11.63236,1.9899999){Tree cap}
\rput(9.232361,-0.0100001525){\textcolor{white}{\textbf{r}}}
\pspolygon[linecolor=black, linewidth=0.04, fillstyle=vlines, 
hatchwidth=0.028222222, hatchangle=0.0, 
hatchsep=0.1411111](2.4323606,-3.6100001)(1.6323606,-2.41)(1.2323605,
-2.8100002)(0.8323605,-2.0100002)(0.032360535,-3.6100001)(0.032360535,
-3.6100001)
\pspolygon[linecolor=black, linewidth=0.04, fillstyle=vlines, 
hatchwidth=0.028222222, hatchangle=0.0, 
hatchsep=0.1411111](4.4323606,-3.6100001)(4.8323607,-2.41)(6.0323606,
-3.6100001)(6.0323606,-3.6100001)
\psframe[linecolor=black, linewidth=0.04, fillstyle=vlines, 
hatchwidth=0.028222222, hatchangle=0.0, hatchsep=0.1411111, 
dimen=outer](10.432361,3.1899998)(9.63236,2.3899999)
\rput(11.63236,2.79){Valid cache }
\psline[linecolor=black, 
linewidth=0.04](9.232361,-1.6100001)(9.232361,-1.6100001)(9.232361,
-1.6100001)(9.232361,-1.6100001)
\psline[linecolor=black, 
linewidth=0.04](9.232361,-0.0100001525)(9.232361,-0.8100002)(9.232361,
-0.8100002)
\pstriangle[linecolor=black, linewidth=0.06, linestyle=dotted, 
dotsep=0.10583334cm, fillstyle=vlines, hatchwidth=0.02, hatchangle=0.0, 
hatchsep=0.2212, dimen=outer](9.232361,-4.01)(4.0,5.2)
\psline[linecolor=black, 
linewidth=0.04](3.2323606,1.9899999)(1.2323605,-0.8100002)(1.2323605,-0.8100002)
\psline[linecolor=black, 
linewidth=0.04](3.2323606,1.1899998)(3.2323606,1.1899998)
\psline[linecolor=black, 
linewidth=0.04](4.8323607,3.59)(4.8323607,3.59)(4.8323607,3.59)(4.8323607,3.59)
\psframe[linecolor=black, linewidth=0.04, fillstyle=solid,fillcolor=colour0, 
dimen=outer](10.432361,2.3899999)(9.63236,1.5899998)
\psframe[linecolor=black, linewidth=0.064, linestyle=dotted, 
dotsep=0.10583334cm, fillstyle=solid, 
dimen=outer](10.432361,3.9899998)(9.63236,3.1899998)
\rput(11.63236,3.59){T(r)}
\end{pspicture}
}

\end{postscript}
\end{center}
\caption{Visualization of definitions for the tree caching problem.}
\label{fig:TreeCacheDefinitions}
\end{figure}

For ease of future algorithm description and readability of proofs we introduce 
some useful definitions. First of all by $T(r)$ we mean subtree of $T$ rooted 
in $r$ and by $h(T)$ height of tree. \textit{Tree cap} is a tree $T_c \subseteq 
T$ such that root $r$ of $T$ belongs to $T_c$ and fact that $v \in T_c$ 
implies, that all the 
nodes on the path from $v$ to $r$ belong to $T_c$. We say that set $C \subseteq 
T$ is \textit{bottom contiguous} if $v \in C$ implies $T(v) \subseteq C$ and we 
call $C$ a \textit{valid cache state}. Notice that this definition does not 
depend on 
cache size, so $C$ can be valid cache state and do not fit into cache size 
at the same time.

The \textbf{TRC} algorithm, after each request, can change the state of 
cache. The only possible changes have to leave the cache in valid state. 
Depending on the character of change (eviction vs. fetch) we can apply on cache 
$C$ either \textit{valid negative changeset} or \textit{valid positive 
changeset}. The valid positive changeset $X$ (think of it as a set that can be 
fetched) is a set for which $X \cap C = \varnothing$ and $C \cup X$ 
is a valid cache state, the negative one (related to eviction) is a set $X 
\subseteq C$ such that $C \ X$ is valid cache state.

We process the input one request by one and assume each request $\sigma(t)$ 
comes in 
time interval $(t-1, t)$. The next request comes in time interval $(t, t + 
1)$ and so on. At time $t$ algorithm changes cache state from $C_{t}$ to 
$C_{t+1}$. Cache state is well defined for round $t$ (which contains time 
interval $(t-1, t)$), because as we see the cache may change only between 
rounds.

Some of the definition which appeared in this section are presented on figure 
~\ref{fig:TreeCacheDefinitions}.

\subsection{Algorithm TRC}

Before finally defining the \textbf{TRC} algorithm we bind to each node $v \in 
T$ a $bank_{t}(v)$ value. At the beginning of algorithm for each $v \in T$ we 
have $bank_{0}(v) = 0$. If node $v$ gets negative or positive request at 
time $t$ for which we pay 1 we increase $bank_{t}(v) = bank_{t-1}(v) + 1$, 
leaving the value for any other node unchanged (so for $v' \neq v 
bank_{t-1}(v') = bank_{t}(v')$). When we change the state of node $v$ (either 
by evicting or fetching it into cache) we clear its bank value (change it to 
$0$). For set of nodes $V 
\subseteq T$, we denote sum $\sum_{v \in V} bank_{t}(v)$ as $bank_{t}(V)$.

The \textbf{TRC} proceeds in phases. First phase starts at time 0 and at the 
start of each phase the \textbf{TRC} cache is empty. 
\begin{algorithm}
\caption{\textbf{TRC}}
\label{alg:TRC}
\begin{algorithmic}[1]
\ForAll{$t \in \{1 \ldots |\sigma|\}$}
  \State Serve request $\sigma(t)$.
  \State Compute $bank_t(v)$ values for $v \in T$.
  \If{exists valid changeset $X$ such that: \begin{itemize}
    \item $bank_t(X) \geq |X| \times \alpha$ (saturation),
    \item $bank_t(Y) < |Y| \times \alpha$ for any valid changeset $Y 
\varsupsetneq X$ (maximality):
\end{itemize}}
 \State Apply $X$.
 \If {cache size $k_{\mathrm{ONL}}$ is exceeded}:
  \State Empty cache.
  \State Reset $bank_t$ values to 0.
  \State Start new phase.
 \EndIf
 \EndIf
\EndFor
  \end{algorithmic}
\end{algorithm}

In line 6 of the algorithm above we do not actually exceed the cache. We 
do not apply last valid changeset, but instead we empty the cache and start new 
phase. 
Every phase ends then whenever applying valid changeset $X$ would lead to cache 
being overloaded except of the last phase: this one can be \textit{unfinished}. 
If $P$ is a phase, additionally by $\mathrm{begin_P}$ and $\mathrm{end_P}$ we 
denote time when the phase $P$ started and finished respectively.

\subsection{Analysis}
Throughout all the analysis we fix the input sequence $I$ and its division into 
phases. We compare \textbf{TRC} with \textbf{OPT} by analyzing both on a single 
phase $P$. By $k_P$ we denote the number of elements in \textbf{TRC's} cache 
after artificial fetch at $\mathrm{end_P}$, before eviction of all cached 
elements.

We start the discussion with some basic properties of changesets. Then we 
introduce concept of
\textit{fields} which will help us proving algorithm's competitiveness.

\subsubsection{Elementary properties}
Remind that $C_t$ denotes the cache state of \textbf{TRC} at round $t$ and that 
any cache change takes place between rounds in discrete times. 
Having that in mind we prove that at any time there cannot be both saturated 
positive and negative changesets.
\begin{lemma}
\label{thm:lemma1}
Fix phase $P$. For any $t \in \{\mathrm{begin_P} +1, \mathrm{begin_P} + 2, 
\ldots, \mathrm{end_P}\}$. 
For any valid changeset $X$ for cache $C_t$ we have $bank_t(X) \leq |X| \cdot 
\alpha$ (1). Moreover changesets $Y$ for $C_t$ for which equality holds are 
all either positive or negative (2).
\end{lemma}
\begin{proof}
We prove this lemma by induction on $t$, starting with $t_1 = \mathrm{begin_P} 
+1$. In 
that time there was only one request, so for all $X$ being valid changeset 
$bank_{t_1} \leq 1$, which proves (1). For (2) see that the request can't be 
both negative and positive.

To process with the induction step, first we prove following property:
\begin{property}
For any changeset $X$ for $C_{t+1}$ we have $bank_t(X) < |X| \cdot \alpha$.
\end{property}
We prove it only for $X$ being positive changeset. For negative ones the proof 
is symmetric. We consider three cases depending on the action taken by 
algorithm \textbf{TRC} at time $t$.
\begin{enumerate}
 \item \textbf{TRC} does not perform any action. That means that there was no 
saturated (saturation is defined in algorithm \ref{alg:TRC} above) changeset 
$X'$ for $C_t$ and $bank_t(X') \cdot \alpha < |X'| \cdot \alpha$ for any $X'$. 
See that $C_t = 
C_{t+1}$, therefore property follows trivially in that case.
  \item \textbf{TRC} fetches $X'$, so $C_{t+1} = C_t \cup X'$. Assume the 
opposite, that there exists changeset $X$ for $C_{t+1}$ such that $bank_t(X) 
\geq |X| \cdot \alpha$. It implies that $X \cup X'$ is positive changeset for 
$C_t$ that might have been fetched instead of $X'$ at time $t$. That 
contradicts the maximality property of changesets applied by algorithm 
\textbf{TRC}.
  \item \textbf{TRC} evicts $X'$, so $C_{t+1}  = C_t \setminus X'$. Assume 
again the opposite. It implies, that $X'' = X \ X'$ is valid 
positive changeset for $C_t$ which is saturated, because $bank_t(X'') = 
bank_t(X) \geq |X| \cdot \alpha \geq |X''| \cdot \alpha$. We have both positive 
and negative saturated changeset for $C_t$ which contradicts induction 
assumption (2).
\end{enumerate}
This property implies the lemma for $t+1$ since we have strict inequalities for 
$C_{t+1}$. When request comes in round $t+1$ it is either positive or negative, 
so if it make a changset saturated it can be either positive or negative one. 
Only one request cames during the round, so strict inequalities can only change 
to equalities, what gives as part (1) of lemma.
\end{proof}
The above lemma convince us that the algorithm is deterministic when choosing 
changeset to apply, cause there is at most one maximal saturated changeset at a 
time.

\subsubsection{Concept of fields}
First notice that whenever there is a positive (negative) request to element 
present (absent) in \textbf(TRC's) cache it neither influence the algorithm 
behaviour nor cost. Therefore from now on we assume that in phase $P$ there 
aren't 
any such requests.
We extend the space of requests from nodes to pairs of node and time $(v, t)$ 
calling them \textit{slots} and we name obtained space \textit{event space}. 
If there 
is a request to node $v$ at time $t$ we say that slot $(v, t)$ is 
\textit{occupied}. We can then see each phase $P$ as a set of occupied slots. 
Figure \ref{fig:spacial_temporal} might help to familiarize with above 
definitions for event space.
\begin{figure}
 \begin{center}
  \includegraphics[width=0.7\textwidth]{spacial_temporal.png}
 \end{center}
 \caption{Event space for node $v$.}
 \label{fig:spacial_temporal}
\end{figure}

As we know what event space and slots are, we can define $fields$. Fields 
form disjoint partition of phase $P$ (partition of slots corresponding 
to $P$). Formally if $X_t$ is changeset applied by \textbf{TRC} at time $t$ we 
define related filed as:
$$F^t = \{(v, t'): v \in X^t \wedge last_v^t < t' \leq t\},$$
where $last_v^{t'}$ denotes the last time strictly before $t'$ in phase $P$ 
when node $v$ changed its state in cache either by being fetched or evicted. If 
there was no such change for $v$ in phase $P$ before $t'$ we assign 
$last_v^{t'} = \mathrm{begin_P}$.
$F^t$ is \textit{positive} if changeset $X_t$ is positive, similar for 
\textit{negative}. We define $req(F^t)$ to be occupied slots from $F^t$. Notice 
that $req(F^t)$ all exactly the requests that 'paid' for changeset $X_t$. 
Together with Lemma \ref{thm:lemma1} we can combine it into simple observation.
\begin{observe}
For field $F^t$ we have $req(F^t) = |X_t| \cdot \alpha$ and the requests 
in one field are either all positive or all negative.
\label{obs:observe1}
\end{observe}
The rest of event space (occupied slots that does not belong to any field $F^t$ 
in phase $P$), we call \textit{open field} and denote by $F^{\infty}$. All 
fields except of 
the open field are denoted by $\mathcal{F}$ and $|\mathcal{F}| = \sum_{F^t \in 
\mathcal{F}} |X^t| \cdot \alpha$.
Additionally for any field $F$, if $A$ is a set of nodes, $F \cap A = \{(v,t) 
\in F: v \in A$ and if $T$ is set of rounds $F \cap T = \{(v, t) \in F: t \in 
T\}$. We introduce another definition related to 
fields: $\tau$\textit{-restricted} subfields of 
$F^t$ which formally are equal
$$F^t_{\leq \tau} = F^t \cap {t': t' \leq \tau}.$$
We can think of it as a snapshot of field taken at time $\tau$ showing which 
slots are occupied up to this time in $F^t$.
\begin{figure}
\begin{center}
  \includegraphics[width=1.1\textwidth]{fields.png}
\end{center}
\caption{Example of fields for tree being a line. $F^1$ is a positive field, 
$F^2$ is negative. $F^{\infty}$ is an open field for presented phase.}
\label{fig:fields}
\end{figure}
Fields are hard to visualize in general case, so we present them on a 
Figure \ref{fig:fields} only for the case when the processed tree structure 
forms a line.

Now we are ready to upper bound the cost of \textbf{TRC} on phase $P$.
\begin{lemma}
Let $\mathcal{F} \cup F^{\infty}$ be a partition of $P$ into fields. We have:
$$cost_{\mathrm{TRC}}(P) \leq 2 \cdot \alpha \cdot |\mathcal{F}| + 
req(F^{\infty}) + k_P \cdot 
\alpha.$$ 
\label{thm:trc_cost}
\end{lemma}
\begin{proof}
The cost on every field $F^t$ originates from serving requests belonging 
to it 
and reorganizing the cache at the end of the field (which ends with fetch or 
eviction)- both this costs are equal 
$|X^t| \cdot \alpha$, which sums up to the first term in right hand side of 
inequality. Expression $req(F^{\infty})$ captures all of the cost associated 
with open field in $P$ and $k_P$ is the cost of final eviction which does not 
occur for the unfinished (last) phase. 
\end{proof}

\subsubsection{Shifting}
Now we process to the main part of our deduction. What we need to do is 
relate the cost of \textbf{OPT} to fields. Just to show intuition behind the 
idea, imagine 
for a moment that requests are evenly distributed within fields. What we mean 
is, every node in every field $F^t$ has exactly $\alpha$ slots 
saturated at time $t$ (so we distribute $|X^t| \cdot \alpha$ requests belonging 
to $F^t$ evenly). It gives us, that every node $v$ alternates between 
periods spent in positive and negative field and on each such period it is 
requested exactly $\alpha$ times. Moreover, we see that on two consecutive 
periods 
(first positive then negative one) \textbf{OPT} pays at least $\alpha$ 
because of changing 
the state of $v$ or paying for all positive or all negative requests. 
Unfortunately, we do not have such fair distribution of requests. What we 
will do is 
we will change phase $P$ to $P'$ which is not harder for \textbf{OPT} but will 
be closer to even distribution of requests. It will give us ability to lower 
bound the cost of \textbf{OPT}.

To achieve such $P'$ we will change $P$ by applying \textit{legal shifts}. 
We say that 
requests placed in slot $(v, t)$ can be legally shifted to slot $(m(v), t)$ if 
$v = m(v)$ or for positive requests $m(v)$ is descendant of $v$ or for negative 
requests $m(v)$ is ascendant of $v$. See, that for fixed sequence of fetches 
and 
removals from cache, for any algorithm, the associated cost can only decrease 
when the legal shifts are applied. For example, consider positive request. 
Shifting it 'down' in the tree can lead to a node which is already in cache and 
we can save on that request. Similar example can be constructed for negative 
requests. We can now make following observation:
\begin{observe}
If $P'$ was created from $P$ by application of legal shifts then 
$cost_{\mathrm{OPT}}(P') \leq cost_{\mathrm{OPT}}(P)$.
\end{observe}

We do not want changing $P$ to $P'$ to drop $cost_{\mathrm{OPT}}$ value too 
much, so we cannot 
argue about it is sufficiently large. What we want to keep is property that 
there are consecutive alternating periods of positive and negative 
requests to node that are sufficiently 'long' in the sense of number of 
requests. 
We will show that we can perform shifts within fields in such way 
that the requests are exactly (for negative requests) or approximately (for 
positive 
requests) evenly distributed, so we can lower bound \textbf{OPT}.

Before we start shifting the requests we present a property of restricted 
subfields and corresponding changesets. For any filed $F$ we say that $S 
\subseteq F$ is \textit{overrequested} if $req(S) > |S| \cdot \alpha$. It turns 
out that some subsets of fields are never overrequested.
\begin{wrapfigure}{l}{0.5\textwidth}
\vspace{-20pt}
 \begin{center}
  \includegraphics[width=0.5\textwidth]{density_pic.png}
\end{center}
\caption{This picture shows the average density of request in fields. The 
darker colour the more request is in average in that part of changeset.}
\vspace{-40pt}
\label{fig:density}
\end{wrapfigure}
\begin{lemma}
Fix $F^t$, $F^t_{\leq \tau}$ with $X^t$ corresponding changeset.
\begin{itemize}
 \item If $F^t$ is negative then for any tree cap $T'$ of $X^t$, $T' \cap 
F^t_{\leq \tau}$ is not overrequested.
\item If $F^t$ is positive then for any subtree $T'\cap F^t_{\leq \tau}$ is not 
overrequested.
\end{itemize}
\label{thm:not_over_requested}
\end{lemma}
\begin{proof}
We prove the property for negative fields only. The proof for positive ones is 
similar. From \ref{thm:lemma1}, because $T' \cap  F^t_{\leq \tau}$ is valid 
changeset at $\tau$, we get $req(T' \cap  F^t_{\leq \tau}) = bank_{\tau}(T' 
\cap F^t_{\leq \tau}) \leq |D \cap F^t_{\leq \tau}| \cdot \alpha$.
\end{proof}
For $\tau = t$ we can apply this lemma and Observation \ref{obs:observe1} 
what gives us a property of fields, intuitively saying that for positive 
fields the saturation of slots in average is more dense closer to 
the root, whereas the density of requests in negative field is higher in the 
'bottom' (see Figure \ref{fig:density}).
\begin{corollary}
Fix $F^t$ filed with corresponding changeset $X_t$. Let $T'$ be tree cap of 
$X_t$, then if the field is:
 \begin{itemize}
  \item positive: $req(F^t \cap T') \geq |T'| \cdot \alpha$,
  \item negative: $req(F^t \cap (X_t \setminus T')) \geq |X_t \setminus 
T'| \cdot \alpha$.
 \end{itemize}
 \label{thm:proper_at_t}
\end{corollary}

\myparagraph{Up-shifting of negative requests}
With up shifting negative requests, we can obtain an even distribution of 
requests across nodes in changeset corresponding to negative field.
\begin{theorem}
By legal shifts of negative requests we can move request within the 
negative field $F^t$ 
and each node has exactly $\alpha$ requests.
\label{thm:legal_shifting_up}
\end{theorem}
To prove this theorem we will first argue that we can always 
shift requests that appeared after first $\alpha$ requests at leaf to the 
parent of that leaf. To make our argument transparent we will say that tree 
cap $Y 
\subseteq X_t$ is \textit{proper} if for any tree cap $Y' \subseteq Y$ and any 
time $\tau \leq t$ it holds that $req(F^t_{\leq \tau} \cap Y') \leq |Y'| \cdot 
\alpha$. Look that Corollary \ref{thm:proper_at_t} tells that at time $t$ 
being end of the field, 
$X_t$ with all its tree caps are proper.

Now we can present a lemma which will be then used as induction step to prove 
Theorem \ref{thm:legal_shifting_up}.
\begin{lemma}
Let $Y$ be a proper tree cap of negative changeset $X_t$ and let $v_l$ be one 
of $Y$'s leaves. If $v_l$ was requested $r_l \geq \alpha$ we can legally shift  
$\delta_l = r_l - \alpha$ requests to parent of $v_l$ ($p(v_l)$), so they are 
kept in $F^t$ and $Y \setminus \{v_l\}$ stays proper. 
\end{lemma}
\begin{proof}
We consider $r_l$ requests that are placed on $v_l$ in chronological order. See 
that when $\alpha + 1$-th request comes, at time $t_l$, $p(v_l)$ has to be in 
cache. If that was not true, we would have overrequested tree cap 
${v_l}$ which 
contradicts Lemma \ref{thm:not_over_requested}. That means, after $\alpha$ first 
requests to $v_l$ the rest $\delta_l$ requests can be moved within field to 
$p(v_l)$.

What remains to be proven is that $Y' = Y \setminus {v_l}$ after
operation described above
is proper. Suppose it is not,so there exists tree cap $D \subseteq Y'$ which 
is overrequested in some time $\tau$. First of all $\tau \geq t_l$ because we 
do not change placement of any request up to time $t_l$ and $Y$ before change 
was not overrequested. Also see that $p(v)$ has to an element of $D$ - we 
did not change the 
number of requests in any other node. After shift $req(F^t \cap Y) = req(F^t 
\cap D) + \alpha$, and if $req(F^t \cap D) > |D| \cdot \alpha$, then $D \cup 
v_l$ would be overrequested contradicting that $Y$ is proper before change.
\end{proof}
Now above lemma, used as induction step, proves Theorem 
\ref{thm:legal_shifting_up}. To see that recall, that 
Corollary \ref{thm:proper_at_t} ensures, that in any proper tree cap $D$ of 
$X_t$ there is a leaf with at least $\alpha$ requests. Going with bottom-up 
manner through the field shifting requests from 'overrequested' nodes leads to 
evenly distributed requests - we leave $\alpha$ requests in every node. That 
completes the proof.   
\myparagraph{Down-shifting of positive requests}
Unfortunately, we can not obtain such even distribution of requests using legal 
shifts in positive fields as in the case of negative fields. Specifically, we 
cannot 
shift legally positive requests down and end up with $\alpha$ requests in every 
node. To see that, imagine we have a tree $T$ with root $r$ and two distinct 
subtrees $T_1$ 
and $T_2$, which both are of size $s$ and have $l$ leaves. Consider following 
series of events (assuming that initially whole tree $T$ is in \textbf{TRC's} 
cache).
\begin{enumerate}
 \item $T_1 \cup \{r\}$ is evicted from cache.
 \item $(s+1) \cdot \alpha - l$ requests to root $r$ appear (no fetch occurs - 
there is not enough requests in $r$ to pay for whole $T_1 \cup \{r\}$).
 \item $T_2$ is evicted.
 \item $s \cdot \alpha$ requests appear at root of $T_1$.
 \item $l$ requests appear at root $r$ of tree $T$. $T$ is again in 
\textbf{TRC's} cache.
\end{enumerate}
\begin{wrapfigure}{r}{0.5\textwidth}
\vspace{-20pt}
 \begin{center}
  \includegraphics[width=0.5\textwidth]{example_not_even.png}
\end{center}
\caption{Numbers indicate chronology of events.}
\vspace{-20pt}
\label{fig:example_not_even}
\end{wrapfigure}
The picture corresponding to this example can be found in Figure 
\ref{fig:example_not_even}. 
Notice that when first $(s+1) \cdot \alpha - l$ request to $r$ appear they can 
only be shifted to $T_1$ nodes. After that root of $T_1$ is given requests 
which can not be moved to $T_2$ - we are unable to do shift up positive 
requests. Therefore only $l$ last requests can be moved to 
$T_2$. We only managed to fill half of the tree nodes with $\Omega(\alpha)$ 
requests.   

Convinced that shifting positive request down is not so easy, our goal is to 
prove that we can at least get, by legal shifting within field $F^t$, 
$\Omega(\frac{|F^t|}{h(T)})$ nodes having $\frac{\alpha}{2}$ 
requests or more. To do that first we argue that from any node $v$ in the 
positive 
field $F^t$ we can move down fraction of its requests so they stay in the field 
and no node $u$ from $T(v)$ gets more then $\frac{\alpha}{2}$ (we do not count 
the requests placed originally at $u$). Notice that to shift any request $(v, 
t_v)$ from $F^t$ to node $v'$ we need to ensure that $v$ is outside the cache, 
what can be expressed by $t_v > last_u(t)$.

\begin{lemma}
Fix $F^t$ positive field with corresponding $X_t$ and $v \in X_t$ that has 
been requested at least $c_v \cdot (\frac{\alpha}{2})$ times for some positive 
natural number $c_v$. There exists legal shifting of $\lceil \frac{\alpha}{2} 
\rceil \cdot (\frac{\alpha}{2})$ requests to nodes 
$T(v) \cap X^t$ chosen from requests sent to $v$ and at least $\lceil 
\frac{\alpha}{2} \rceil$ of them is given 
$\frac{\alpha}{2}$. All these shifts are within the field $F^t$.
\label{thm:request_mapping}
\end{lemma}
\begin{proof}
We sort the elements of $T(v) \cap X_t$ with respect to increasing time of last 
eviction (function $last_v$), getting sequence of nodes $\{u_i\}_{i \geq 
0}$. Since $v$ has to be one of the first elements that are evicted from cache 
in that positive field, we assume $u_0 = v$. We sort the request sequence to 
$v$ too and get the sequence with increasing arrival time. The second 
sequence will be denoted by $\{(v, t_j)\}_{j=0}^{c_v \cdot \frac{\alpha}{2} - 
1}$. We define partial function \textit{sh} on latter sequence's 
elements, that will say to which element of $T(v) \cap X_t$ request should be 
pushed down. Specifically, $sh((v, t_{k \cdot \alpha + m})) = u_k$, where $m$ 
values are from $[0, \frac{\alpha}{2})$. The function $sh$ is not defined on 
the rest of requests. We will not push those requests down. The 
mapping definition is presented on Figure \ref{fig:req_map}.
\begin{figure}
 \begin{center}
  \includegraphics[width=1.1\textwidth]{request_mapping.png}
 \end{center}
 \caption{Request mapping from Lemma \ref{thm:request_mapping}. The requests 
are divided into alternating parts of size $\lceil \frac{\alpha}{2} \rceil$ 
and $\lfloor \frac{\alpha}{2} \rfloor$. The odd fragments (counting from $1$) 
are mapped, whereas even parts are omitted in mapping}
 \label{fig:req_map}
\end{figure}

Now we need to prove that defined shifting keeps requests within the field, 
so 
for shifted requests $t_j \geq last_{sh((v, t_j))}(t)$. Assume that it is not 
true and $t_j < last_{sh((v, t_j))}(t)$ for $j = k \cdot \alpha + m$ 
corresponding to shifted request. Since request are mapped with respect to 
increasing order of sequence $\{u\}$, $|F^t_{\leq t_j}| \cap T(v)| <= k|$, but 
the same time $req(F^t_{\leq t_j}| \cap T(v)) = j$ and $F^t_{\leq t_j}| \cap 
T(v)$ is then overrequested. This with Lemma \ref{thm:not_over_requested} leads 
to contradiction.
\end{proof}
We are ready to prove the main lemma of this paragraph, which gives us 
possibility to obtain even distribution of positive request to the factor 
$O(\frac{1}{h(T)})$.
\begin{lemma}
For any positive field $F^t$ it is possible to shift 
legally $\frac{req(F^t)}{2}$ requests within the field, so at least 
$\frac{|X^t|}{2 \cdot h(T)}$ nodes in $F^t$ have 
$\frac{\alpha}{2}$ requests or more.
\label{thm:legal_shifting_down}
\end{lemma}
\begin{proof}
Let $X^t$ be corresponding changeset for $F^t$. We take all the requests in 
every node and arrange them into groups of size $\frac{\alpha}{2}$ 
chronologically, meaning, that first $\frac{\alpha}{2}$ requests to node are in 
first 
group and so on. If the last group is not full (it has less then  
$\frac{\alpha}{2}$ requests) then it is omitted. The number of request that are 
not omitted in set $X$ will be denoted by $\overline{req}(X)$. Notice that  
$\overline{req}(F^t) \geq |X^t| \cdot \frac{\alpha}{2}$, since $req(F^t) = 
|X^t| \cdot \alpha$ and we omit at most $\frac{\alpha}{2}$ requests at every 
node.
 \begin{figure}
 \begin{center}
  \includegraphics[width=0.5\textwidth]{layers.png}
\end{center}
\caption{Division of requests to layers. Requests to nodes, which have the same 
distance from $X^t$'s root are on the same level.}
\label{fig:layers}
\end{figure}

We partition the elements of $X^t$. In each part there are elements with the 
same distance from root of $X^t$. We call the parts levels. The nodes that are 
distance $i$ from root are on level $i$ and are denoted $X^t_i$. We 
have exactly $h(T)$ layers. By pigeonhole principle there exists a level 
$X^t_i$ for which $\overline{req}(F^t \cap X^t_i) \geq \frac{req(F^t)}{2} \cdot 
h(T)$ (look at Figure \ref{fig:layers}). The subtrees rooted in nodes from 
$X^t_i$ are distinct and to each of them we can apply Lemma 
\ref{thm:request_mapping}. Node $v \in X^t_i$ has $\overline{req}(F^t \cap 
\{v\})$ requests so from the lemma we we can shift $\frac{\alpha}{2}$ of 
them to $\lceil \overline{req}(F^t \cap \{v\} \rceil$ distinct nodes in $T(v)$. 
Summing up the 'filled' nodes for each element from $X^t_i$, we get at least 
$\lceil \overline{req}(F^t \cap X^t_i) \rceil \geq \frac{|X^t|}{2 \cdot h(T)}$ 
with $\frac{\alpha}{2}$ requests.
\end{proof}
\subsubsection{Competitiveness of TRC}
We will use results from previous subsection to lower bound \textbf{OPT's} cost 
and finally getting the main theorem of this thesis about competitive ratio of 
\textbf{TRC}. This proofs are mostly technical, but still intuitions from 
previous sections might be useful to understand idea behind them.

We will need one auxiliary lemma about \textbf{TRC's} cache at the end of the 
phase. We remind that this last cache state is special - its size exceeds 
$k_{\mathrm{ONL}}$, because instead of normal fetch we perform artificial one 
described earlier.  
\begin{lemma}
Fix phase $P$ and $C$ cache state at the end of the phase ($\mathrm{end_P}$). 
Any tree 
cap $D \subseteq C$ received at least $|D| \cdot \alpha$ requests during phase 
$P$.
\label{thm:lots_of_req_in_tc_end_of_p}
\end{lemma}
\begin{proof}
Let $X_1, X_2, \ldots, X_m$ be all positive chengesets from phase $P$. We make 
separation on them, defining $X'_i = X_i \setminus \bigcup_{j=1}^{i-1} X_j$. 
Note 
that $X'_i$ is tree cap of $X_i$. The sequence $X'_1, X'_2, \ldots, X'_m$ is 
pairwise disjoint and covers $C$, so $C \subseteq \bigsqcup_{j=1}^m 
X'_j$. Let $X_j^D = D \cap X'_j$, then $D = \bigsqcup_{j=1}^m X_j^D$ since $D 
\subseteq C$. $X_j^D$ is tree cup of $X'_j$, so by transitivity of relation of
being a tree cap, $X_j^D$ makes up a tree cap of $X_j$. From Corollary 
\ref{thm:proper_at_t} we have that $req(X_j^d \cap F_i) \geq |X_j^D| \cdot 
\alpha$. There can be even more requests on the whole phase - not only 
restricted to 
field $F_i$. Since $X_j^D$ are pairwise disjoint, we can sum up requests over 
all $j$ what proves the lemma.
\end{proof}

\myparagraph{Lower bound for OPT}
Throughout this paragraph we use a notation of $V_{\mathrm{OPT}}$ - it will be 
the set of nodes 
that appeared in \textbf{OPT's} cache during phase.$V_{\mathrm{OPT}}^C$ will 
denote the complement to the whole tree $T$ ($T \setminus V_{\mathrm{OPT}}$). 
Notice, 
that $V_{\mathrm{OPT}}$ creates a proper cache state.

We get two different lower bounds for \textbf{OPT} presented in the next two 
lemmas. 
\begin{lemma}
For any phase $P$, $cost_{\mathrm{OPT}}(P) \geq (k_P - k_{\mathrm{OPT}}) \cdot 
\alpha$.
\label{thm:opt_bound_with_k}
\end{lemma}
\begin{proof}
Let $C$ be the cache state of \textbf{TRC} at $\mathrm{end_P}$ (after 
artificial fetch, 
so it has $k_P$ elements in cache). We consider separately the cost 
\textbf{OPT} obtained on the nodes $V_{\mathrm{OPT}}$ and $V_{\mathrm{OPT}}^c$. 
For the first 
set, \textbf{OPT} had to fetch at least $|V_{\mathrm{OPT}}| - k_{\mathrm{OPT}}$ 
nodes during the 
phase paying $\alpha$ for each of them. For the second one, see that 
$C \setminus V_{\mathrm{OPT}}$ is a subset of $V_{\mathrm{OPT}}^c$. Moreover, 
$C \setminus 
V_{\mathrm{OPT}}$ is an union of disjoint tree caps of subtrees belonging to 
$C$. From 
Lemma \ref{thm:lots_of_req_in_tc_end_of_p} all those tree caps are dense, so 
summing requests in them gives us at least $|C \setminus V_{\mathrm{OPT}}| 
\cdot 
\alpha$ requests to nodes in $C \setminus V_{OPT}$ which belongs to 
$V_{\mathrm{OPT}}^c$ 
as mentioned above. Joining results for both cases, $cost_{\mathrm{OPT}}(P) 
\geq 
(|V_{\mathrm{OPT}}| - k_{\mathrm{OPT}}) \cdot \alpha + |C \setminus 
V_{\mathrm{OPT}}| \cdot 
\alpha = (|C| - k_{\mathrm{OPT}}) \cdot \alpha$ what ends the prove since $k_P$ 
is 
defined as $|C|$ after artificial fetch. 
\end{proof}
\begin{lemma}
For any phase $P$, $cost_{\mathrm{OPT}}(P) \geq (\frac{|\mathcal{F}|}{4 \cdot 
h(T)} 
-k_P) \cdot \frac{\alpha}{2}$.
\label{thm:opt_bound_with_F}
\end{lemma}
\begin{proof}
Let $P$ be a fixed phase. We perform shifting on $P$ so that Lemma 
\ref{thm:legal_shifting_up} and Lemma \label{thm:legal_shifting_down} hold. The 
result of this action will be called phase $P'$. From the previous discussion 
it is sufficient to prove that $cost_{\mathrm{OPT}}(P') \geq 
(\frac{|\mathcal{F}|}{4 
\cdot h(T)}-k_P) \cdot \frac{\alpha}{2}$. To do that we first focus on each $v$ 
node separately. See that it's history can be divided into alternating 
intervals when it was out of the cache and was charged by receiving the positive 
requests or it was in the cache charged by negative requests. We call those 
periods $OUT$ and $IN$ respectively. Cache is empty in the beginning of the 
phase, so every node starts with $OUT$ period. Two consecutive $OUT$ and $IN$ 
periods form \textit{stripe}. Notice, that for every node $v$ only the last 
$OUT$ period in phase $P'$ might be unpaired, therefore there is at most $k_P$ 
unpaired periods.

Let's denote by $cnt_{\mathrm{IN}}$ and $cnt_{\mathrm{OUT}}$ the number of all 
$IN$ and $OUT$ 
periods in $P$ respectively. From the above discussion, $cnt_{\mathrm{OUT}} = 
cnt_{\mathrm{IN}} + 
k_P$. The sum of all periods (both $IN$ and $OUT$) is equal to the sum of sizes 
of all fields (excluding the open one) which we denoted by $|\mathcal{F}|$. It 
implies that $|\mathcal{F}| = cnt_{\mathrm{OUT}} + cnt_{\mathrm{IN}}$ so 
$cnt_{\mathrm{OUT}} = 
\frac{|\mathcal{F}| + k_P}{2} \geq \frac{|\mathcal{F}|}{2}.$

Now we will use the lemmas about shifting. From them we know that in $P'$ all 
$IN$ periods consist of $\alpha$ requests and there is at least 
$\frac{1}{2h(T)}$ throughout $OUT$ periods paired with $IN$ periods that have 
at least $\frac{\alpha}{2}$ requests. Therefore there are at least 
$(\frac{cnt_{\mathrm{OUT}}}{2h(T)} - k_P)$ pairs of $OUT$ and $IN$ periods that 
both 
have $\frac{\alpha}{2}$ requests or more. On each such pair \textbf{OPT} pays 
at least $\frac{\alpha}{2}$ because of changing nodes state in cache (at cost 
$\alpha$) or paying from either positive or negative requests from that stripe. 
Stripes are distinct, so the cost of \textbf{OPT} can be bounded from below by 
$(\frac{cnt_{\mathrm{OUT}}}{2h(T)} - k_P) \cdot \frac{\alpha}{2} \geq 
(\frac{|\mathcal{F}|}{4h(T)} - k_P) \cdot  \frac{\alpha}{2}$.
\end{proof}

\myparagraph{Upper bound for open fields}
To prove the competitiveness of \textbf{TRC} we have to deal with open fields. 
Lemma below solves this problem.
\begin{lemma}
For any phase $P$ and its open field $F^{\infty}$, $req(F^{\infty}) \leq 
(k_{\mathrm{ONL}} + k_{\mathrm{OPT}}) \cdot \alpha + 2 \cdot 
cost_{\mathrm{OPT}}(P)$.
\label{thm:inifinite_filed_bound}
\end{lemma}
\begin{proof}
Let $X_{\mathrm{end_P}}$ be the	artificially applied changeset at time 
$\mathrm{end_P}$. We divide open field $F^{\infty}$ into two disjoint sets. 
First of them contains positive requests, the second: negative. Formally
\begin{gather*}
F^{\infty}_{+} = \{(v, t): v \notin C_{\mathrm{end_P}} \cup 
X_{\mathrm{end_P}}, t \geq last_v(\mathrm{end_P})\}, \\
F^{\infty}_{-} = \{(v, t): v \in C_{\mathrm{end_P}}, t \geq 
last_v(\mathrm{end_P})\}. 
\end{gather*}
Now we divide requests from $F^{\infty}$ into three distinct sets 
$F^{\infty}_{-}$, $F^{\infty}_{+} \cap V_{\mathrm{OPT}}$ and $F^{\infty}_{+} 
\cap V_{\mathrm{OPT}}^c$. We will bound the requests in each of them 
separately. We can do so, since the above division forms partition covering  
$F^{\infty}$. Therefore $req(F^{\infty}) = req(F^{\infty}_{-}) + 
req(F^{\infty}_{+} \cap V_{\mathrm{OPT}}) + req(F^{\infty}_{+} \cap 
V_{\mathrm{OPT}}^c)$.

Let's start with bounding $F^{\infty}_{-}$. Since its nodes are all in the 
cache $C_{\mathrm{end_P}}$, the number of nodes in $F^{\infty}_{-}$ does not 
exceed $k_{\mathrm{ONL}}$. These nodes were not evicted from the cache, which 
means, that number of requests in $req(F^{\infty}_{-})$ is at most 
$k_{\mathrm{ONL}} \cdot \alpha$.

Now see, that $req(F^{\infty}_{+} \cap V_{\mathrm{OPT}}^c) \leq 
req(V_{\mathrm{OPT}}^c)$. The nodes from $V_{\mathrm{OPT}}^c$ were never in 
\textbf{OPT's} cache during phase $P$, so it has to pay for all of them. this 
gives us $req(V_{\mathrm{OPT}}^c) \leq cost_{\mathrm{OPT}}(P)$ and finally 
$req(F^{\infty}_{+} \cap V_{\mathrm{OPT}}^c) \leq cost_{\mathrm{OPT}}(P)$.

To bound the last set, notice that $F^{\infty}_{+}$ is a valid changeset for 
$C_{\mathrm{end_P}} \cup X_{\mathrm{end_P}}$. The same time $V_{\mathrm{OPT}}$ 
forms a valid cache state. Combining this two observations, we conclude that 
$F^{\infty}_{+} \cap V_{\mathrm{OPT}}$ is a valid changeset for 
$C_{\mathrm{end_P}} \cup X_{\mathrm{end_P}}$ (and they intersect empty). Let 
$X^{V}_{+}$ be the set of nodes in $F^{\infty}_{+} \cap V_{\mathrm{OPT}}$. If 
$req(F^{\infty}_{+} \cap V_{\mathrm{OPT}})$ would be greater then $|X^{V}_{+}| 
\cdot \alpha$ then we can extend changeset $X_{\mathrm{end_P}}$ by these nodes 
contradicting the maximality property of changesets applied by \textbf{TRC}. 
Therefore $req(F^{\infty}_{+} \cap V_{\mathrm{OPT}}) \leq |X^{V}_{+}| \cdot 
\alpha \leq |V_{\mathrm{OPT}}| \cdot \alpha \leq k_{\mathrm{OPT}} \cdot \alpha 
+ (|V_{\mathrm{OPT}}| - k_{\mathrm{OPT}}) \leq k_{\mathrm{OPT}} \cdot \alpha 
+ cost_{\mathrm{OPT}}(P)$.

Now we can combine the inequalities discussed above.
\begin{multline*}
req(F^{\infty}) = req(F^{\infty}_{-}) + 
req(F^{\infty}_{+} \cap V_{\mathrm{OPT}}) + req(F^{\infty}_{+} \cap 
V_{\mathrm{OPT}}^c) \\
\leq (k_{\mathrm{ONL}} \cdot \alpha) + (k_{\mathrm{OPT}} \cdot \alpha 
+ cost_{\mathrm{OPT}}(P)) + (cost_{\mathrm{OPT}}(P)) \\ 
= (k_{\mathrm{ONL}} + 
k_{\mathrm{OPT}}) \cdot \alpha + 2 \cdot 
cost_{\mathrm{OPT}}(P)
\end{multline*}
\end{proof}
\myparagraph{Main result}
Now we are ready to state and prove main result.
\begin{theorem}
Algorithm \textbf{TRC} is $O(\frac{h(T) \cdot k_{\mathrm{ONL}}}{k_{\mathrm{ONL}} 
- k_{\mathrm{OPT}} + 1})$-competitive.
\label{thm:main_theorem}
\end{theorem}
\begin{proof}
It is sufficient to show that for any phase $P$, 
$\frac{cost_{\mathrm{TRC}}}{cost_{\mathrm{ONL}}} = O(\frac{h(T) \cdot 
k_{\mathrm{ONL}}}{k_{\mathrm{ONL}} - k_{\mathrm{OPT}} + 1})$. Then summing up 
the cost over all the phases gives the desired result.

We fix phase $P$. Recall, that from Lemma \ref{thm:trc_cost} we know that 
$cost_{\mathrm{TRC}}(P)$ is bounded by sum $2 \cdot \alpha \cdot |\mathcal{F}| 
+ req(F^{\infty}) + k_P \cdot \alpha$. We bound the first two summands using 
the inequalities from previously proved lemmas. Firstly, from Lemma 
\ref{thm:opt_bound_with_F} we can derive, that $|\mathcal{F}| \leq 
4h(T) \cdot (cost_{\mathrm{OPT}}(P) + k_P \cdot \alpha)$. Secondly, from Lemma 
\ref{thm:inifinite_filed_bound} $req(F^{\infty}) \leq 
(k_{\mathrm{ONL}} + k_{\mathrm{OPT}}) \cdot \alpha + 2 \cdot 
cost_{\mathrm{OPT}}(P) \leq 2 \cdot (k_P \cdot \alpha + 
cost_{\mathrm{OPT}}(P))$. Therefore $2 \cdot \alpha \cdot |\mathcal{F}| + 
req(F^{\infty}) = O(h(T)) \cdot (k_P \cdot \alpha + cost_{\mathrm{OPT}}(P))$.
\end{proof}

\myparagraph{Lower bound for competitive ratio}
It turns out the competitive ratio, which is a result of Theorem 
\ref{thm:main_theorem} is optimal up to factor $O(h(T))$. We prove this 
statement by showing that any online algorithm has to be at least 
$\Omega (\frac{k_{\mathrm{ONL}}}{k_{\mathrm{ONL}}- k_{\mathrm{OPT}} + 1})$.
\begin{theorem}
For any $\alpha$ being a cost of changing one element's state in cache, any 
online algorithm for tree caching problem is at least $\Omega 
(\frac{k_{\mathrm{ONL}}}{k_{\mathrm{ONL}}- k_{\mathrm{OPT}} + 1})$-competitive.
\end{theorem}
\begin{proof}
To prove this theorem we will use Theorem 5 from publication \cite{tarjan}. The 
result from the article stays, that any online algorithm solving paging problem 
(move from slow memory to fast memory costs $1$ and there is no bypassing) is 
at least $\Omega(\frac{k_{\mathrm{ONL}}}{k_{\mathrm{ONL}}- k_{\mathrm{OPT}} + 
1})$-competitive. The cache sizes are defined the same as for the tree caching 
problem. What is sufficient to do is to show a bijective mapping from any 
online algorithm for paging problem to an online algorithm for tree caching and 
mapping from any input for paging problem to an input for tree caching 
problem. We can loose only constant factor on cost when performing both 
algorithms on corresponding inputs.

We equate all pages $p$ from paging problem fixing a bijection $f$ with leaves 
$l \in T$, so $f(p) = l$. Let's define the mapping between inputs. By $I$ 
denote the input for paging and by $I'$: input for tree caching problem. We 
change $I$ sequence going through its requests one by one. Whenever there is 
request to $p$ we add to $I'$ exactly $\alpha$ requests to $f(p)$. Now we would 
like to argue that for any online algorithm for paging problem $A$ that pays 
$c$ on input $I$, we can construct algorithm $A'$ for tree caching problem, that 
on input $I'$ pays $\theta(c \cdot \alpha)$. We define $A'$ as follows. 
When $A$ serves request $r$ from $I$ and fetches $r$ to its cache, algorithm 
$A'$ bypasses corresponding $\alpha$ requests to $f(r)$ and fetches $r$. If $A$ 
evicts some pages, $A'$ evicts corresponding leaves. Notice, that $A'$ has the 
same cache state as $A$ if we apply $f$ on its elements and look at its state 
every $\alpha$ (for $A$ - every one) requests. Therefore, if $A$ does not pay 
for some request, then $A'$ does not pay for it neither. $A'$ pays then at most 
$O(\alpha)$ times more then $A$ (all eviction operations' cost on the sequence 
$I'$ has asymptotically the same value as all fetch operations' cost).

We can inverse the reduction above and see that $A$ pays on $I$ is 
$O(\frac{1}{\alpha})$ times the cost of $A'$.  
  
\end{proof}
