\section{Introduction}
\textit{Online algorithms} theory is an approach to study problems in 
interactive computing. The general idea is that the input is coming to a 
system not  whole at one time but in portions at different time. The system 
goal is to serve those inputs elements in portions, not knowing what will the 
future input look like. The system has to result in valid output and stay in 
legal state. Moreover, we have some cost function associated with serving piece 
of input in defined system state. Cost usually mirrors the amount of time 
needed for performing an operation or some other resources usage. The 
algorithms aims to minimize this cost.

One of the online algorithm problems is caching problem, whose variant is 
studying in this thesis. In caching (paging) problem we need to maintain 
two-level memory system that consists of fast memory which is small (the reason 
it is small originates from the practical application - fast memory is more 
expensive) and slow memory which is large enough to deal with the input 
sequence. The input consists of elements that should be fetched from memory. If 
the element is already in the fast memory the basic setting of problem lets us 
handle it without any cost. Otherwise, we have to fetch the element from slow 
memory paying non-zero cost. Sometimes it is required to place any requested 
element immediately in cache. If there is no place in fast memory (cache) for 
the element, we have to move some items from fast memory back to slow memory. 
If such requirement does not exist and we can serve item without putting it to 
the cache (obtaining some big cost instead) we say that the model is 
\textit{with bypassing}.

It is not hard to see many application of the paging problem. We can map it to 
the memory management problem in computer operating systems. Here slow memory 
is disc and the fast memory is RAM. Another example is web browsers cache which 
store some previous responses from the server in fast memory and is able to 
respond some queries much faster.

In this thesis we consider new setting of caching problem. First of all, 
elements that are requested are not independent. What we mean is, that the 
space of items forms a tree and whenever we have node $v$ in fast memory, we 
are obligated to keep whole subtree rooted in $v$ in cache, too. Moreover, our 
model differs from the previously studied ones by introducing two types of 
requests - \textit{positive} and \textit{negative}. For the positive request we 
pay when we serve it and it is not in cache (so as in usual setting), weather 
for the negative one we pay when it is in the cache.

This model finds its application in handling Forwarding Information Base (FIB). 
It is a set of rules which IP routers have to store to forward request 
correctly. The number of such rules is dramatically increasing in modern 
routers and the memory storing FIB is known to be expensive and to need a lot 
of energy. Technology of Software-Defined Networking (SDN) introduces the idea 
of keeping 'most popular' FIB rules in faster and more expensive memory. 
Clearly, FIB forms a tree, in that way solving our model is strongly connected 
to improving SDN performance.

We present an online algorithm \textbf{TRC} to solve the caching problem 
for presented model. To measure its performance we use the \textbf{competitive 
analysis}. Its basic idea is to compute cost ratio (which is called 
\textit{competitive ratio}) between cost of online algorithm to cost of 
\textit{offline algorithm}, which knows whole the input sequence from the 
beginning. The main result presented is online algorithm \textbf{TRC}, whose 
competitive ratio we prove is $O(\frac{height(T) \cdot k_{ONL}}{k_{ONL} - k_{OPT} 
+ 1})$.

In section \ref{caching_problems} we review competitive analysis basic 
definitions. We also give brief survey of online algorithm and their analysis 
for standard paging problem. We consider both deterministic and randomized 
solutions for this problem. In section \ref{tree_caching_algo} we present 
deterministic algorithm \textbf{TRC} for our model of tree caching with 
dependencies and two kinds of requests. Then we prove the theorem about its 
competitive ratio and show lower bound on competitive ratio for any online 
algorithm solving our problem. \textbf{TRC} turns out to be optimal up to the 
factor $O(h(T))$, where $h(T)$ is the hight of input tree. The last section 
gives a more precise description of motivation and importance of tree caching 
problem.
