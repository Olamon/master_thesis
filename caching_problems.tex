\section{Caching problems}
\label{caching_problems}
In this section, we present an overview of caching problems. The basic paging 
problem setting is well studied from the competitive analysis perspective. 
Both deterministic and randomized optimal solutions are already known. We focus 
here on deterministic algorithms, although we give a sketch of randomized solutions 
because of their popularity and importance in online algorithm theory as a 
whole. 
\subsection{Basic definitions}
We shall remind ourselves about basic concepts of online algorithms theory. 
Let $\sigma = \sigma(1), \sigma(2), \ldots, \sigma(n)$ be a sequence of 
requests. An \textit{online algorithm} $A$ has to process requests one by one, 
without any knowledge of the future requests. Formally, when 
serving request 
$\sigma(t), 1 \leq t \leq n,$ the algorithm does not know any request 
$\sigma(t')$ for $t'>t$. Handling any request incurs $cost$ whose value  
depends on the request and algorithm state. The goal is to minimize the overall 
$cost$.

Sleator and Tarjan \cite{tarjan} propose
\textit{competitive analysis} as a way to evaluate the performance of online 
algorithms. In this approach, we compare \csigma{A}, the cost of online algorithm 
$A$ on input $\sigma$, to \csigma{OPT}, the cost of the optimal offline algorithm, 
which knows all the input sequence beforehand. We call an algorithm 
\textit{r-competitive} if there exists a constant $c$ such that
$$C_{\mathrm{A}}(\sigma) \leq r \times C_{\mathrm{OPT}}(\sigma) + c$$
for any request sequence $\sigma$. We say that $r$ is a \textit{competitive ratio}
of an algorithm if it is a minimal value, such that the algorithm is 
\textit{r-competitive}. The goal is therefore to minimize $r$.

\subsection{Caching and model variants}
The caching problem can be described as follows. We are given two types of memory: 
fast memory, which has size $k$, and slow memory (secondary storage) with size $N 
\gg k$. We can 
assume that the number of requests $n$ fulfills $N \geq n \gg k$. When getting a 
request $\sigma_{e}$ for an item $e$ that is already in the cache, we serve it 
without any cost (or at the cost $c \ll 1$). 
Otherwise, we have to retrieve $e$ from the slow memory and pay 
$1$. Sometimes, whenever $e$ is not in the cache, we require that before 
serving $\sigma_{e}$, we must place $e$ in the cache, and therefore must have 
enough free space in fast memory to move it there. If there is no such requirement 
(and $\sigma_{e}$ can 'omit' moving into cache) we say that the model is 
\textit{with bypassing}. Usually, bypassing the cache is very expensive. After serving 
a request, the algorithm can change its cache by evicting and fetching elements 
from secondary storage. Some cost is assigned to these operations, as well.

In the tree caching problem that we study, a requested element $v$ can depend 
on some other elements. Specifically, it depends on its descendants such that 
when we decide to fetch $v$ we have to fetch all nodes from $T(v)$ (maximal subtree of $T$ rooted in $v$ with respect to set inclusion)
that are not already in the cache.

There is another generalization of the caching problem that we mostly focus on in 
this thesis. It is called 
$(k_{\mathrm{ONL}}, k_{\mathrm{OPT}})$\textit{-paging generalization}. It gives 
the online and optimal offline algorithms different resources: the online cache has
size $\kind{ONL}$, while the optimal offline cache has size 
$k_{\mathrm{OPT}} \leq k_{\mathrm{ONL}}$. It might seem 
unfair to give the optimal solution less space, but sometimes it can give us 
some information about how resources can influence the performance. Remember 
that the offline algorithm has much more power as it knows all the future.

\subsection{Deterministic algorithms}
We review well-known algorithms that solve the caching problem without bypassing, 
where every requested item has to be placed in the cache and any reference to 
the slow memory has unit cost 1.
\subsubsection{Longest Forward Distance (LFD) - offline optimal solution}
Consider following \textit{lazy} (evicts a page if and only if it has no space for a new one) 
offline algorithm: 
\begin{myalgo}
Whenever there is a cache miss (an item requested is not 
present in fast memory), evict the element whose next request is the latest among other 
cached elements.
\end{myalgo}

This algorithm is called \textbf{Longest Distance Forward} (LDF), and it turns 
out that it is the optimal offline solution. It is very uncommon to know the 
optimal offline solution, and this knowledge really helps in competitive analysis.
Moreover, this algorithm has polynomial 
time complexity, whereas most online theory problem are not solvable in 
polynomial time. The optimality of \textbf{LFD} was proven by Belady 
\cite{LFDBelady}.
\begin{theorem} 
 \textbf{LFD} is the optimal offline algorithm for paging.
\end{theorem}
\begin{proof}
 The proof is based on applying the following lemma (its proof can be found in 
\cite{czarodziej}).
\begin{lemma}
Let $A$ be any offline algorithm for paging, $\sigma$ - request sequence. For 
any $i \in \{1 \ldots |\sigma|\}$ we can construct an offline algorithm $A_i$ such 
that:
\begin{itemize}
 \item[(a)] $A_i$ serves first $i-1$ items the same as $A$,
 \item[(b)] If the $i$th request results in a page fault, $A_i$ evicts from cache 
items in the cache which will be latest requested (with 'longest distance 
forward'),
  \item[(c)] $A_i(\sigma) \leq A(\sigma)$.
\end{itemize}
\end{lemma}
Having this lemma, we can prove the theorem by applying it $|\sigma|$ times. 
We start with any $OPT$ algorithm and apply the lemma, so we get $OPT_1$. Then 
we apply the lemma to $OPT_1$ for $i=2$ getting $OPT_2$, and so on. It is easy to 
see that $OPT_{|\sigma|}$ acts like \textbf{LFD} and pays cost at most the 
same 
as $OPT$.
\end{proof}

\subsubsection{Marking algorithms examples (LRU, FWF)}
Now we will consider two deterministic algorithms.
\begin{myalgo}
 \textbf{Least Recently Used (LRU)}
 \newline
When eviction is necessary, evict item whose most recent request was the 
earliest among cached items.
\end{myalgo}
\begin{myalgo}
  \textbf{Flush When Full (FWF)}
  \newline
Whenever there is a page fault and the cache is full, evict all elements from the cache.
\end{myalgo}
Both of these algorithms belong to a class of algorithms called \textit{marking}. 
They are characterized by associating a bit to each page called
its mark. Initially all pages are unmarked. 

We divide the input sequence $\sigma$ into $k$\textit{-phases} dependent on value $k$ 
(cache size). We define the $i$th phase as a maximal input sequence, following 
phase $i-1$ such that it contains at most $k$ different items requested. So, 
excluding the last phase, each phase ends when the next request $r$ would 
increase the number of distinct pages in the phase to $k+1$. A useful observation is 
that $r$ has to be \textit{new}, meaning it is different then any item requested 
in phase $i$.
\begin{figure}
\begin{postscript}
\psscalebox{0.8. 0.8}
{
\begin{pspicture}(0,-1.42)(16.918556,1.42)
\definecolor{colour0}{rgb}{0.78431374,0.78431374,0.78431374}
\psframe[linecolor=black, linewidth=0.04, dimen=outer](15.62,0.6)(0.02,-0.6)
\psframe[linecolor=black, linewidth=0.04, dimen=outer](1.22,0.6)(0.02,-0.6)
\psframe[linecolor=black, linewidth=0.04, dimen=outer](2.42,0.6)(1.22,-0.6)
\psframe[linecolor=black, linewidth=0.04, dimen=outer](3.62,0.6)(2.42,-0.6)
\psframe[linecolor=black, linewidth=0.04, dimen=outer](4.82,0.6)(3.62,-0.6)
\psframe[linecolor=black, linewidth=0.04, dimen=outer](6.02,0.6)(4.82,-0.6)
\psframe[linecolor=black, linewidth=0.04, fillstyle=solid,fillcolor=colour0, 
dimen=outer](7.22,0.6)(6.02,-0.6)
\psframe[linecolor=black, linewidth=0.04, fillstyle=solid,fillcolor=colour0, 
dimen=outer](8.42,0.6)(7.22,-0.6)
\psframe[linecolor=black, linewidth=0.04, fillstyle=solid,fillcolor=colour0, 
dimen=outer](9.62,0.6)(8.42,-0.6)
\psframe[linecolor=black, linewidth=0.04, fillstyle=solid,fillcolor=colour0, 
dimen=outer](10.82,0.6)(9.62,-0.6)
\psframe[linecolor=black, linewidth=0.04, dimen=outer](12.02,0.6)(10.82,-0.6)
\psframe[linecolor=black, linewidth=0.04, dimen=outer](13.22,0.6)(12.02,-0.6)
\psframe[linecolor=black, linewidth=0.04, dimen=outer](14.42,0.6)(13.22,-0.6)
\psframe[linecolor=black, linewidth=0.04, fillstyle=solid,fillcolor=colour0, 
dimen=outer](15.62,0.6)(14.42,-0.6)
\psdots[linecolor=black, dotsize=0.2](16.02,-0.6)
\psdots[linecolor=black, dotsize=0.2](16.42,-0.6)
\psdots[linecolor=black, dotsize=0.2](16.82,-0.6)
\psline[linecolor=black, 
linewidth=0.04](0.02,-1.0)(0.02,-1.4)(6.02,-1.4)(6.02,-1.0)(6.02,-1.0)
\psline[linecolor=black, 
linewidth=0.04](6.02,1.0)(6.02,1.4)(10.82,1.4)(10.82,
1.0)(10.82,1.0)
\psline[linecolor=black, 
linewidth=0.04](10.82,-1.0)(10.82,-1.4)(14.42,-1.4)(14.42,-1.0)(14.42,-1.0)
\psline[linecolor=black, 
linewidth=0.04](14.42,1.0)(14.42,1.4)(16.02,1.4)(16.02,1.4)
\rput(2.82,-1.0){PHASE 1}
\rput(8.42,1.0){PHASE 2}
\rput(12.82,-1.0){PHASE 3}
\rput[bl](0.42,-0.2){1}
\rput[bl](1.62,-0.2){2}
\rput[bl](2.82,-0.2){1}
\rput[bl](4.02,-0.2){3}
\rput[bl](5.22,-0.2){2}
\rput[bl](6.42,-0.2){4}
\rput[bl](7.62,-0.2){2}
\rput[bl](8.82,-0.2){1}
\rput[bl](10.02,-0.2){2}
\rput[bl](11.22,-0.2){3}
\rput[bl](12.42,-0.2){2}
\rput[bl](13.62,-0.2){1}
\rput[bl](14.82,-0.2){5}
\psline[linecolor=black, linewidth=0.04, arrowsize=0.05291666666666667cm 
2.0,arrowlength=1.4,arrowinset=0.0]{<-}(6.82,-0.2)(6.82,-1.0)(6.82,-1.0)
\rput[bl](6.82,-1.4){element k+1}
\end{pspicture}
}
\end{postscript}
\caption{Example of sequence devided to phases for cache size $k=3$.}
\label{fig:PhasesDeterministic}
\end{figure}

In any marking algorithm, during a phase, we mark a page when it is first 
requested in the phase. Finally, in the phase time a marking algorithm
never evicts a page, that was marked during the phase. Now we will show a theorem 
characterizing the class of marking algorithms.
\begin{theorem}\cite{torng}
For any marking algorithm $A_{\mathrm{mark}}$ with cache of size 
$\kind{ONL}$, if $OPT$ 
cache size is $\kind{OPT} \leq \kind{ONL}$, $A_{\mathrm{mark}}$ is 
$O(\frac{\kind{OPT}}{\kind{ONL}-\kind{OPT}+1})$-competitive.
\end{theorem}
\begin{proof}
Fix any input sequence $\sigma$ and divide it into $\kind{ONL}$-phases as described 
above. $A_{\mathrm{mark}}$ on any such phase pays at most $\kind{ONL}$ (because there are 
at most 
$\kind{ONL}$ distinct requests in one phase). To bound the $OPT$ cost, we will bound it with 
respect to \textit{shifted phases}. A shifted phase $i$ consists 
of the first element of phase $i+1$ (call it $f_{i+1}$) and all requests from phase 
$i$th except the first one (call it $f_i$). Figure 
~\ref{fig:PhaseDeterministic} shows the phases and shifted phases for $\kind{ONL}=3$. Observe, 
that to obtain the shifted phase window we moved the phase window by one element. At 
the beginning of $i$-th shifted phase $OPT$ has to have $f_i$ in its cache. 
Therefore, it can have at most $\kind{OPT}-1$ distinct elements, that are different then $f_i$ before shifted phase $i$ starts. 
A shifted phase has at least $\kind{ONL}$ distinct elements therefore $OPT$ has at least 
$\kind{ONL}-\kind{OPT}+1$ page faults during the shifted phase (there can be at most two shifted 
phases at the end, that do not have this property, and we omit them because the cost can be put 
into the constant when comparing costs).
Summing up costs for $A_{\mathrm{mark}}$ by phases and $OPT$ by shifted phases 
we get:
$$ A_{\mathrm{mark}}(\sigma) \leq \frac{\kind{ONL}}{\kind{ONL}-\kind{OPT}+1} \cdot OPT(\sigma) + const,$$
where we put to the $const$ value all the costs obtained from requests not covered by phases (or 
shifted phases). This cost is bounded, since the number of not covered phases and shifted phases is at most 3.
\end{proof}
Now we should prove that both \textbf{FWF} and \textbf{LRU} are
marking algorithms.
\begin{propo}
 \textbf{FWF} and \textbf{LRU} are marking algorithms.
\end{propo}
\begin{proof}
 We fix the request sequence $\sigma$ and consider its $\kind{ONL}$-phase division 
(where $\kind{ONL}$ is the cache size of any of the two online algorithms).

First, see that \textbf{FWF} is a marking algorithm. At the start of each phase, 
its cache is empty and it does not evict any pages during whole phase. 
Therefore, it certainly does not evict any marked page.

Now suppose that \textbf{LRU} is not a marking algorithm. Then there exists a 
phase $p$ and an element $x$ that is marked during this phase and evicted before 
the phase ends. Assume that both $p$ and $x$ are the first such phase and element. 
Consider the moment when $x$ is marked (the first time it was requested in $p$). 
For that time it is the most recently requested element. For the \textbf{LRU} to 
evict $x$, it has to be 'the oldest' element in the cache, and therefore there must be 
$\kind{ONL}$ elements distinct from $x$ requested. The eviction takes place 
during $p$, so it must contain 
at least $\kind{ONL}+1$ distinct elements, which contradicts with the $\kind{ONL}$-phase definition. 
\end{proof}
This gives us that both algorithms are $O(\frac{\kind{ONL}}{\kind{ONL}-\kind{OPT}+1})$-competitive. Not 
only marking algorithms obtain this ratio. One example of such an algorithm is 
\textbf{First-in First-out} (FIFO), which evicts a page that was in the cache for 
the longest time.

\subsubsection{Non-competitive algorithms examples}
In this part we consider two well known strategies, that turn out to be not 
competitive at all.
\begin{myalgo}
 \textbf{Last-in First-out (LIFO)}
 \newline
If removal is necessary, evict the element which was moved to the cache most 
recently.
\end{myalgo}
\begin{myalgo}
 \textbf{Least Frequently Used (LFU)}
 \newline
If removal is necessary, remove the element which was least frequently requested since the start of its 
presence in the cache.
\end{myalgo}

To prove that those algorithms are not competitive, it is sufficient to show a sequence 
of requests that potentially incurs infinite cost when compared to the cost of 
\textbf{OPT}, which is finite. In the case of both algorithms presented above, we 
will use only 
$k+1$ different pages, denoted $p_1, p_2, \ldots, p_{k+1}$, where $k$ is the 
size of the cache.

We start with \textbf{LIFO}. Suppose that at the beginning \textbf{LIFO} holds
elements $p_1, p_2, \ldots p_k$ in its cache (we can always renumber the 
elements, so we can 
assume that). Consider the following sequence of requests:
$$ \sigma = (p_1, p_2, \ldots, p_k, p_{k+1}, p_k, p_{k+1}, \ldots).$$
The idea is that we request the elements that \textbf{LIFO} just removed from the 
cache. See that on every request after the first $p_{k+1}$ requests this algorithm pays 1, 
so the total cost tends to infinity, whereas \textbf{OPT}, by keeping $p_k$ 
and $p_{k+1}$ in its cache, gets at most one page fault on this subsequence starting at the first element $p_{k+1}$. 
This proves the non-competitiveness of \textbf{LIFO}.

In the case of \textbf{LFU}, fix any constant integer $c$ and sequence:
$$\sigma = (p_1^c, p_2^c, \ldots, p_{k-1}^c, (p_k, p_{k+1})^{c-1}),$$
where the superscript means how many times we repeat the request. We can see that 
after the first $(k-1) \cdot c + 1$ requests \textbf{LFU} will pay for each of the 
next requests, so it pays $O(c)$ for all the sequence. On the other hand, 
\textbf{OPT} will pay for only one page fault when $p_{k+1}$ occurs for the 
first time, keeping both $p_k$ and $p_{k+1}$ in its cache. Since we can fix $c$ to be any value and we can repeat this 
sequence many times as well, we can get an infinite sequence $\sigma$ for which 
$\frac{\csigma{LFU}}{\csigma{OPT}} = c \rightarrow \infty$. 
\textbf{LFU} is then not competitive.
\subsection{Randomized algorithms (RAND, MARK)}
When we analyzed deterministic algorithms, we had only one model of 
\textit{adversary}. His task was to figure out a sequence of requests to 
maximize the competitive ratio of online algorithms. Looking at randomized 
algorithms, the considered adversary can be given less or more power. To clarify,
when describing an adversary we should answer a few questions: Does he have to 
provide all of the input sequence explicitly or can he generate the sequence in the online fashion? 
Does he know the algorithm and randomization used by the online algorithm? With these questions 
in mind, we define different adversaries that the randomized algorithm may deal with.
Usually we recognize three types of adversaries
(see also \cite{bbktw}):
\begin{itemize}
\item \textbf{Oblivious Adversary:} The adversary has to show the entire request 
sequence before any of the elements are processed by the online algorithm. The online algorithm is known to 
the adversary. The adversary's cost is the cost of optimal offline algorithm on the 
sequence. Notice that this adversary is similar to the one used when analyzing
deterministic algorithms.
\item \textbf{Adaptive Online Adversary:} This adversary can observe the online 
algorithm and generate the next request using knowledge about the previous (randomized) 
answers of the online algorithm. It has to serve requests online though, without 
knowing the online algorithm's actions on the present or any future requests.
\item \textbf{Adaptive Offline Adversary:} This adversary is the same as above except for the 
way of being charged: it obtains cost that equals the optimal offline algorithm's cost
on the generated sequence.
\end{itemize}
The competitive ratio for a randomized algorithm is dependent on the type of 
adversary we choose. The easiest definition is for the oblivious adversary. We say 
that an online algorithm $A$ is $c$-competitive against the oblivious adversary if, for 
all input sequences $\sigma$ generated by this adversary, we have:
$$\mathbb{E}[cost_{\mathrm{A}}(\sigma)] \leq c\cdot cost_{\mathrm{OPT}}(\sigma) 
+ const,$$
where the expected value is taken over the random choices of $A$. For adaptive 
adversaries, on the other hand, we say $A$ is $c$-competitive if:
$$\mathbb{E}[cost_{\mathrm{A}}(\sigma) - c \cdot cost_{\mathrm{ADV}}(\sigma)]
\leq const.$$
Here the expected value is still taken over the random choices of $A$. The difference 
is that the $cost_{\mathrm{ADV}}$ is a random variable as well and depends on the randomization 
of $A$. 
In the case of the adaptive offline adversary, at least we know that 
$cost_{\mathrm{ADV}}(\sigma)$ and $cost_{\mathrm{OPT}}(\sigma)$ have the same 
distribution, whereas for the online 
adaptive adversary such a simple nature of the cost's random variable is not known.

It turns out that randomization can't give an improvement on the competitive 
ratio in the case of an adaptive offline adversary. The following theorem was 
presented in \cite{power}:
\begin{theorem}
If there is a $c$-competitive algorithm against an adaptive offline adversary, then 
there exists a $c$-competitive deterministic online algorithm. 
\end{theorem}
The relation between adversaries is captured in the same paper and presented in the 
theorem below.
\begin{theorem}
Suppose $A$ is a $c$-competitive randomized algorithm against any adaptive online 
adversary and additionally there exists a $d$-competitive algorithm against any 
oblivious adversary. Then $A$ is $c \cdot d$-competitive against an adaptive 
offline adversary.
\end{theorem}
Usually we call the oblivious adversary 'weak', the adaptive online adversary 'medium', and 
the adaptive offline adversary 'strong'. These informal names are consistent with our intuition: 
the oblivious adversary is the easiest to compete with, whereas the adaptive offline adversary is 
the hardest, as it is the most powerful.

Now we present a randomized algorithm \textbf{MARK}, which turns out to be 
$2 \cdot H_k$-competitive against the oblivious adversary. $H_k$ is the called $k$th 
\textit{harmonic number} and is defined as $H_k = \sum^k_{i=1} \frac{1}{i}$. 
This series has a really nice property that $\ln k < H_k \leq \ln k + 1$. 
Algorithm \textbf{MARK} is a marking algorithm, so it has a bit associated with 
each of the $k$ entries in cache.

\begin{algorithm}
\caption{\textbf{MARK}}
\label{alg:MARK}
\begin{algorithmic}[1]
\ForAll{$r$ in sequence $\sigma$}
  \If{$r$ is in cache}
    \State Mark $r$.
  \Else
    \If{Cache is full}
      \If{All elements are marked}
	\State Unmark all cached elements.
      \EndIf
      \State Choose $p$ uniformly at random from unmarked pages.
      \State Remove $p$ from cache.
    \EndIf
    \State Fetch $r$ and mark it.
  \EndIf
\EndFor
\end{algorithmic}
\end{algorithm}
\begin{theorem}
\cite{markcom} Algorithm \textbf{MARK} is $2 \cdot H_k$-competitive against the oblivious 
adversary.
\end{theorem}
\begin{proof}
We fix an input sequence $\sigma$ and its $k$-phases partition defined the same as 
in the case of deterministic marking algorithm. We can assume that both 
\textbf{OPT} and \textbf{MARK} start with the same cache state (adjusting the 
cache can be done with constant cost not influencing the competitive ratio).

Consider the $i$th phase. We call \textbf{old pages} all the elements requested 
in 
phase $i$, that were also requested in phase $i-1$. All the other pages requested 
in phase $i$ are called \textit{new}. For the first phase, all the pages are new. We 
denote the number of new pages for the phase $i$ by $m_i$. A picture of these 
definitions can be found at Figure \ref{fig:PhasesRandomized}.
\begin{figure}
\begin{postscript}
% \usepackage[usenames,dvipsnames]{pstricks}
% \usepackage{epsfig}
% \usepackage{pst-grad} % For gradients
% \usepackage{pst-plot} % For axes
\psscalebox{0.75 0.75} % Change this value to rescale the drawing.
{
\begin{pspicture}(0,-2.0524511)(16.918556,2.0524511)
\definecolor{colour0}{rgb}{0.78431374,0.78431374,0.78431374}
\psframe[linecolor=black, linewidth=0.04, 
dimen=outer](15.62,1.2324512)(0.02,0.03245117)
\psframe[linecolor=black, linewidth=0.04, 
dimen=outer](1.22,1.2324512)(0.02,0.03245117)
\psframe[linecolor=black, linewidth=0.04, 
dimen=outer](2.42,1.2324512)(1.22,0.03245117)
\psframe[linecolor=black, linewidth=0.04, 
dimen=outer](3.62,1.2324512)(2.42,0.03245117)
\psframe[linecolor=black, linewidth=0.04, 
dimen=outer](4.82,1.2324512)(3.62,0.03245117)
\psframe[linecolor=black, linewidth=0.04, 
dimen=outer](6.02,1.2324512)(4.82,0.03245117)
\psframe[linecolor=black, linewidth=0.04, fillstyle=solid,fillcolor=colour0, 
dimen=outer](7.22,1.2324512)(6.02,0.03245117)
\psframe[linecolor=black, linewidth=0.04, fillstyle=hlines*,fillcolor=colour0, 
hatchwidth=0.028222222, hatchangle=0.0, hatchsep=0.2412, 
dimen=outer](8.42,1.2324512)(7.22,0.03245117)
\psframe[linecolor=black, linewidth=0.04, fillstyle=solid,fillcolor=colour0, 
dimen=outer](9.62,1.2324512)(8.42,0.03245117)
\psframe[linecolor=black, linewidth=0.04, fillstyle=hlines*,fillcolor=colour0,
hatchwidth=0.028222222, hatchangle=0.0, hatchsep=0.2412, 
dimen=outer](10.82,1.2324512)(9.62,0.03245117)
\psframe[linecolor=black, linewidth=0.04, 
dimen=outer](12.02,1.2324512)(10.82,0.03245117)
\psframe[linecolor=black, linewidth=0.04, fillstyle=hlines, 
hatchwidth=0.028222222, hatchangle=0.0, hatchsep=0.2412, 
dimen=outer](13.22,1.2324512)(12.02,0.03245117)
\psframe[linecolor=black, linewidth=0.04, 
dimen=outer](14.42,1.2324512)(13.22,0.03245117)
\psframe[linecolor=black, linewidth=0.04, fillstyle=solid,fillcolor=colour0, 
dimen=outer](15.62,1.2324512)(14.42,0.03245117)
\psdots[linecolor=black, dotsize=0.2](16.02,0.03245117)
\psdots[linecolor=black, dotsize=0.2](16.42,0.03245117)
\psdots[linecolor=black, dotsize=0.2](16.82,0.03245117)
\psline[linecolor=black, 
linewidth=0.04](0.02,-0.36754882)(0.02,-0.7675488)(6.02,-0.7675488)(6.02,
-0.36754882)(6.02,-0.36754882)
\psline[linecolor=black, 
linewidth=0.04](6.02,1.6324512)(6.02,2.0324512)(10.82,2.0324512)(10.82,
1.6324512)(10.82,1.6324512)
\psline[linecolor=black, 
linewidth=0.04](10.82,-0.36754882)(10.82,-0.7675488)(14.42,-0.7675488)(14.42,
-0.36754882)(14.42,-0.36754882)
\psline[linecolor=black, 
linewidth=0.04](14.42,1.6324512)(14.42,2.0324512)(16.02,2.0324512)(16.02,
2.0324512)
\rput(2.82,-0.36754882){PHASE 1}
\rput(8.42,1.6324512){PHASE 2}
\rput(12.82,-0.36754882){PHASE 3}
\rput[bl](0.42,0.43245116){1}
\rput[bl](1.62,0.43245116){2}
\rput[bl](2.82,0.43245116){1}
\rput[bl](4.02,0.43245116){3}
\rput[bl](5.22,0.43245116){2}
\rput[bl](6.42,0.43245116){4}
\rput[bl](7.62,0.43245116){2}
\rput[bl](8.82,0.43245116){5}
\rput[bl](10.02,0.43245116){2}
\rput[bl](11.22,0.43245116){3}
\rput[bl](12.42,0.43245116){4}
\rput[bl](13.62,0.43245116){1}
\rput[bl](14.82,0.43245116){5}
\rput[bl](6.42,-1.5675489){element k+1}
\psline[linecolor=black, linewidth=0.04, arrowsize=0.05291666666666668cm 
2.0,arrowlength=1.4,arrowinset=0.0]{<-}(6.42,0.43245116)(7.22,-1.1675488)(7.22,
-1.1675488)
\psline[linecolor=black, linewidth=0.04, arrowsize=0.05291666666666668cm 
2.0,arrowlength=1.4,arrowinset=0.0]{<-}(8.02,0.43245116)(10.42,-1.5675489)(10.42
,-1.5675489)
\psline[linecolor=black, linewidth=0.04, arrowsize=0.05291666666666668cm 
2.0,arrowlength=1.4,arrowinset=0.0]{<-}(10.42,0.43245116)(10.42,
-1.5675489)(10.42,-1.5675489)
\psline[linecolor=black, linewidth=0.04, arrowsize=0.05291666666666668cm 
2.0,arrowlength=1.4,arrowinset=0.0]{<-}(12.42,0.43245116)(10.42,
-1.5675489)(10.42,-1.5675489)
\rput[bl](9.62,-1.9675488){old elements}
\end{pspicture}
}
\end{postscript}
\caption{Example of sequence divided to phases for cache size $k=3$. Old pages 
are indicated by the hatched squares. The number of new pages in the first phase
$m_1 = 3$ and $m_2 = m_3 = 2$ or the second and the third phase.}
\label{fig:PhasesRandomized}
\end{figure}


We can see that old pages are somehow beneficial for \textbf{MARK}. If it keeps 
an old page sufficiently long (up to the moment when it is requested 
in phase $i$ for the first time), it does not pay for this page throughout the whole phase. Seeing that 
leads us to conclude that the worst case is when all the new 
$m_i$ elements are requested first followed by the old pages so that the probability of an old 
page still being in the cache is the smallest possible. More specifically, the 
probability that the j$th$ old page (they are ordered by their first 
occurrence in the phase) is still in the cache when it is requested can be bounded from below by 
$\frac{k-m_i-(j-1)}{k-(j-1)}$, since $k-m_i-(j-1)$ is the smallest possible number of the old pages in 
the cache that are unmarked and $k-(j-1)$ is the biggest possible number of the unmarked pages 
up to the time the $j$th old page is requested for the first time. This gives us that 
the probability of a page fault on that request is equal to $1 - 
\frac{k-m_i-(j-1)}{k-(j-1)} = \frac{m_i}{k-(j-1)}$, so the expected number of 
page faults in the whole phase equals (recall that we pay $1$ for every new page):
$$ m_i + \sum_{j=1}^{k - m_i}\frac{m_i}{k-(j-1)} = m_i\cdot(1+H_k-H_{m_i}) \leq 
m_i \cdot H_k.$$

Now we have to lower bound the cost of \textbf{OPT}. Consider two consecutive 
phases $i-1$ and $i$. There are $k + m_i$ distinct pages requested during these two 
phases. Since the \textbf{OPT} cache size is $k$, there are at least $m_i$ page faults. 
During the first phase, \textbf{OPT} has at least $m_1$ faults. Summing up the 
number of faults for all
the pairs of phases $(i-1, i)$ we get that \textbf{OPT} has at least $\sum m_i$ page faults. Since each phase 
is counted at most twice in this sum
(except the first one, which is counted once), the lower bound for the 
\textbf{OPT} cost is $\frac{1}{2} \cdot \sum m_i$. As we know, \textbf{MARK's} 
number of page faults, when we take sum over the phases, is at most $H_k \cdot \sum 
m_i$, which proves the theorem.
\end{proof}
