\begin{figure}
\begin{postscript}
% \usepackage[usenames,dvipsnames]{pstricks}
% \usepackage{epsfig}
% \usepackage{pst-grad} % For gradients
% \usepackage{pst-plot} % For axes
\psscalebox{0.75 0.75} % Change this value to rescale the drawing.
{
\begin{pspicture}(0,-1.81)(16.918556,1.81)
\definecolor{colour0}{rgb}{0.78431374,0.78431374,0.78431374}
\psframe[linecolor=black, linewidth=0.04, dimen=outer](15.62,0.99)(0.02,-0.21)
\psframe[linecolor=black, linewidth=0.04, dimen=outer](1.22,0.99)(0.02,-0.21)
\psframe[linecolor=black, linewidth=0.04, dimen=outer](2.42,0.99)(1.22,-0.21)
\psframe[linecolor=black, linewidth=0.04, dimen=outer](3.62,0.99)(2.42,-0.21)
\psframe[linecolor=black, linewidth=0.04, dimen=outer](4.82,0.99)(3.62,-0.21)
\psframe[linecolor=black, linewidth=0.04, dimen=outer](6.02,0.99)(4.82,-0.21)
\psframe[linecolor=black, linewidth=0.04, fillstyle=solid,fillcolor=colour0, 
dimen=outer](7.22,0.99)(6.02,-0.21)
\psframe[linecolor=black, linewidth=0.04, fillstyle=solid,fillcolor=colour0, 
dimen=outer](8.42,0.99)(7.22,-0.21)
\psframe[linecolor=black, linewidth=0.04, fillstyle=solid,fillcolor=colour0, 
dimen=outer](9.62,0.99)(8.42,-0.21)
\psframe[linecolor=black, linewidth=0.04, fillstyle=solid,fillcolor=colour0, 
dimen=outer](10.82,0.99)(9.62,-0.21)
\psframe[linecolor=black, linewidth=0.04, dimen=outer](12.02,0.99)(10.82,-0.21)
\psframe[linecolor=black, linewidth=0.04, dimen=outer](13.22,0.99)(12.02,-0.21)
\psframe[linecolor=black, linewidth=0.04, dimen=outer](14.42,0.99)(13.22,-0.21)
\psframe[linecolor=black, linewidth=0.04, fillstyle=solid,fillcolor=colour0, 
dimen=outer](15.62,0.99)(14.42,-0.21)
\psdots[linecolor=black, dotsize=0.2](16.02,-0.21)
\psdots[linecolor=black, dotsize=0.2](16.42,-0.21)
\psdots[linecolor=black, dotsize=0.2](16.82,-0.21)
\psline[linecolor=black, 
linewidth=0.04](0.02,-0.61)(0.02,-1.01)(6.02,-1.01)(6.02,-0.61)(6.02,-0.61)
\psline[linecolor=black, 
linewidth=0.04](6.02,1.39)(6.02,1.79)(10.82,1.79)(10.82,1.39)(10.82,1.39)
\psline[linecolor=black, 
linewidth=0.04](10.82,-0.61)(10.82,-1.01)(14.42,-1.01)(14.42,-0.61)(14.42,-0.61)
\psline[linecolor=black, 
linewidth=0.04](14.42,1.39)(14.42,1.79)(16.02,1.79)(16.02,1.79)
\rput(2.82,-0.61){PHASE 1}
\rput(8.42,1.39){PHASE 2}
\rput(12.82,-0.61){PHASE 3}
\rput[bl](0.42,0.19){1}
\rput[bl](1.62,0.19){2}
\rput[bl](2.82,0.19){1}
\rput[bl](4.02,0.19){3}
\rput[bl](5.22,0.19){2}
\rput[bl](6.42,0.19){4}
\rput[bl](7.62,0.19){2}
\rput[bl](8.82,0.19){1}
\rput[bl](10.02,0.19){2}
\rput[bl](11.22,0.19){3}
\rput[bl](12.42,0.19){2}
\rput[bl](13.62,0.19){1}
\rput[bl](14.82,0.19){5}
\rput[bl](7.62,-1.01){element k+1}
\psline[linecolor=black, linewidth=0.04, arrowsize=0.05291666666666667cm 
2.0,arrowlength=1.4,arrowinset=0.0]{<-}(6.82,0.59)(8.42,-0.61)(8.42,-0.61)
\psline[linecolor=black, linewidth=0.04, linestyle=dashed, dash=0.17638889cm 
0.10583334cm](1.22,-0.61)(1.22,-1.41)(7.22,-1.41)(7.22,-0.61)(7.22,-0.61)
\rput[b](4.02,-1.81){SHIFTED PHASE 1}
\end{pspicture}
}
\end{postscript}
\caption{Example of sequence devided to phases for cache size $k=3$.}
\label{fig:PhaseDeterministic}
\end{figure}
